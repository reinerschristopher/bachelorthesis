%----------------------------------------------------------------------------------------
%	PACKAGES AND OTHER DOCUMENT CONFIGURATIONS
%----------------------------------------------------------------------------------------

\documentclass[
11pt, % The default document font size, options: 10pt, 11pt, 12pt
%oneside, % Two side (alternating margins) for binding by default, uncomment to switch to one side
ngerman,
singlespacing, % Single line spacing, alternatives: onehalfspacing or doublespacing
%nolistspacing, % If the document is onehalfspacing or doublespacing, uncomment this to set spacing in lists to single
%liststotoc, % Uncomment to add the list of figures/tables/etc to the table of contents
%toctotoc, % Uncomment to add the main table of contents to the table of contents
%parskip, % Uncomment to add space between paragraphs
headsepline, % Uncomment to get a line under the header
%chapterinoneline, % Uncomment to place the chapter title next to the number on one line
%consistentlayout, % Uncomment to change the layout of the declaration, abstract and acknowledgements pages to match the default layout
]{MastersDoctoralThesis} % The class file specifying the document structure

\usepackage[latin1]{inputenc} % Required for inputting international characters. [utf8/applemac/latin1]
\usepackage[T1]{fontenc} % Output font encoding for international characters
% \usepackage[ngerman]{babel}
% \usepackage{luaotfload}
% \usepackage[EU2]{fontenc}
% \usepackage{lmodern}

% \usepackage{mathpazo} % Use the Palatino font by default
% \usepackage{mathpple}
% \usepackage{eulervm}
% \usepackage{concrete}
% \usepackage[charter]{mathdesign}
% \usepackage{newtxtext,newtxmath}
% \usepackage{newpxmath}
% \usepackage{lmodern}


\usepackage[backend=biber,style=numeric,natbib=true]{biblatex} % Use the bibtex backend with the numeric citation style, backend=bibtex

\addbibresource{quellen.bib} % The filename of the bibliography

\usepackage[autostyle=true]{csquotes} % Required to generate language-dependent quotes in the bibliography

\usepackage{siunitx} % Number formatting. [group-separator={.}]
\sisetup{group-separator = {\,}, group-minimum-digits=5}  % small spacing between every third digit of long number
% https://www.namsu.de/Extra/pakete/Siunitx.html

\setcounter{secnumdepth}{3}

\usepackage{lineno}
\linenumbers
\modulolinenumbers[5]

%----------------------------------------------------------------------------------------
%	OWN SETTINGS
%----------------------------------------------------------------------------------------

\setcounter{tocdepth}{3}  % A value of X=0 means that your table of contents will show nothing at all and 5 means, that even subparagraphs will be shown.

% Define some commands to keep the formatting separated from the content 
\newcommand{\keyword}[1]{\textbf{#1}}
\newcommand{\tabhead}[1]{\textbf{#1}}
\newcommand{\code}[1]{\texttt{#1}}
\newcommand{\file}[1]{\texttt{\bfseries#1}}
\newcommand{\option}[1]{\texttt{\itshape#1}}

%----------------------------------------------------------------------------------------
%	MARGIN SETTINGS
%----------------------------------------------------------------------------------------

\geometry{
	paper=a4paper, % Change to letterpaper for US letter
	inner=2.5cm, % Inner margin
	outer=3.8cm, % Outer margin
	bindingoffset=.5cm, % Binding offset
	top=1.5cm, % Top margin
	bottom=1.5cm, % Bottom margin
	%showframe, % Uncomment to show how the type block is set on the page
}

%----------------------------------------------------------------------------------------
%	THESIS INFORMATION
%----------------------------------------------------------------------------------------

\thesistitle{Analyse des UMAP Verfahrens} % Your thesis title, this is used in the title and abstract, print it elsewhere with \ttitle
\author{Christopher Reiners} % Your name, this is used in the title page and abstract, print it elsewhere with \author

\supervisor{Prof. Dr. Jochen Garcke}
\subject{Mathematik}
\thesistype{Bachelorarbeit}
\university{Rheinischen Friedrich-Wilhelms-Universit�t Bonn}
\department{Mathematisches Institut f�r Numerische Simulation} 
\faculty{Mathematisch-Naturwissenschaftliche Fakult�t der}
\geburtsdatum{9. April 1998}  %Author DoB
\geburtsort{Detmold}  %Author place of birth
\date{6. August 2019}  %\today  %Abgabedatum
\zweitgutachter{Dr. Bastian Bohn}



% \AtBeginDocument{
% \hypersetup{pdftitle=\thesistitle} % Set the PDF's title to your title
% \hypersetup{pdfauthor=\author} % Set the PDF's author to your name
% \hypersetup{pdfkeywords=\keywordnames} % Set the PDF's keywords to your keywords
% }

\begin{document}

\frontmatter % Use roman page numbering style (i, ii, iii, iv...) for the pre-content pages

\pagestyle{plain} % Default to the plain heading style until the thesis style is called for the body content

%----------------------------------------------------------------------------------------
%	TITLE PAGE
%----------------------------------------------------------------------------------------

\fontfamily{cmr}\selectfont
\maketitle

%----------------------------------------------------------------------------------------
%	QUOTATION PAGE
%----------------------------------------------------------------------------------------

% \vspace*{0.2\textheight}

% \noindent\enquote{\itshape Do not worry about your difficulties in Mathematics. I can assure you mine are still greater.}\bigbreak

% \hfill Albert Einstein

%----------------------------------------------------------------------------------------
%	ACKNOWLEDGEMENTS
%----------------------------------------------------------------------------------------

\begin{acknowledgements}

\addchaptertocentry{\acknowledgementname} % Add the acknowledgements to the table of contents

An dieser Stelle m�chte ich gerne Prof. Dr. Jochen Garcke, f�r die Vergabe dieses sehr interessanten 
und spannenden Themas, danken. Besonderer Dank gilt Leland McInnes f�r das pers�nliche Gespr�ch in Ottawa, Kanada. 
So konnte ich die Motivation, welche er f�r das UMAP Verfahren hatte, aus erster Hand erfahren.

Gerne m�chte ich auch meinen Freunden danken, welche mich im Laufe der Studienzeit, besonders 
w�hrend der Bachelorarbeit, begleitet haben. Anna f�r die vielen Eispausen und die lustigen Unterhaltungen 
auch sp�t in der Nacht, Tobi, Hendrik und Lennard f�r die Motivation und den mathematischen Austausch und Lukas 
f�r die hilfreichen Gespr�che. Ohne euch w�re diese Arbeit nicht entstanden. 

Zum Schluss danke ich meiner Familie f�r Ihre bedingungslose Unterst�tzung, trotz Einsilbigkeit meiner Antworten 
in den letzten Wochen meiner Arbeit.

\end{acknowledgements}

%----------------------------------------------------------------------------------------
%	LIST OF CONTENTS/FIGURES/TABLES PAGES
%----------------------------------------------------------------------------------------

\tableofcontents % Prints the main table of contents

%\listoffigures % Prints the list of figures

%\listoftables % Prints the list of tables

%----------------------------------------------------------------------------------------
%	THESIS CONTENT - CHAPTERS
%----------------------------------------------------------------------------------------

\mainmatter % Begin numeric (1,2,3...) page numbering

\pagestyle{thesis} % Return the page headers back to the "thesis" style

%!TEX root = ../main.tex

\chapter{Einleitung} 

\label{Einleitung}

% Relevanz, �bersicht, UMAP, Struktur

%----------------------------------------------------------------------------------------

%Define math commands used in this chapter
\newcommand{\R}{\mathbb{R}}  % Symbol for real numbers
\newcommand{\seqx}{\{x_i\}_{i=1}^N, \ (x_i \in \mathbb{R}^D)}  % Sequence of x_i
\providecommand{\abs}[1]{\lvert#1\rvert}  % produces | x |
\providecommand{\norm}[1]{\lVert#1\rVert}  % produces || x ||

%----------------------------------------------------------------------------------------

% N O T I Z E N
% 

% T E X T B A U S T E I N E
% 'UMAP wird f�r single cell RNA-sequencing genutzt'
% scRNA hat mehr rauschen als bulkRNA-seq und wird zur Untersuchung der genetischen 
% Information (Genexpression) einzelner Zellen genutzt.
% 'Dimension ist Anzahl der Freiheitsgerade'
% Es gibt F�lle in denen es unm�glich ist die gesamte globale Struktur zu erhalten. 
% Betrachet man beispielsweise gleichverteilte Punkte auf einem Kreis und m�chte diese 
% in R einbetten, muss es zwei Punkte geben welche getrennt werden.

%----------------------------------------------------------------------------------------
%----------------------------------------------------------------------------------------

% TODO: Was macht das Thema Dimensionsreduktion relevant?

%----------------------------------------------------------------------------------------

\section{Datenanalyse}
% Einordnung verschiedener DR Algorithmen. (Geometrie, Stochastik, Topologie)
% Was ist eine gute Repr�sentation der Daten?
% Sollte unabh�ngig von Transformationen weniger wichtiger Eigenschaften sein und 
% unterscheidbar bzgl. relevanter Eigenschaften. (evtl. Bengio zitieren)
% Fr�her 'feature engineering' jetzt lernen der wichtigen Eigenschaften (siehe Bengio 2013)
% Welche Formen von Datenanalyse (statistische, geometrische,...)

%-----------------------------------

% \subsection{Topologische Datenanalyse}
Eine neuere Form der Datenanalyse - die topologische Datenanalyse (\textit{kurz: TDA}) - 
nutzt mathematische Instrumente des Teilgebiets der Topologie 
(\textit{griechisch: Lehre vom Ort/Platz}) um Daten zu beschreiben und zu strukturieren.

Wir beschreiben die Stituation einen Datensatz $X$ zu analysieren wie folgt. Wir betrachten 
einen $D$-dimensionalen Raum und ein Objekt $K$. In unserem Fall werden wir uns gr��tenteils 
mit dem euklidischen Raum $\R^D$ versehen mit der euklidischen Norm $\norm{\cdot}$ besch�ftigen.

$K$ kann beispielsweise eine geschlossene Menge sein. Die genaue Struktur von $K$ bleibt uns allerdings unbekannt. 
Sp�ter werden wir argumentieren, dass $K$ gewisse Regularit�tseigenschaften erf�llt und 
in vielen F�llen lokal einem niedrigdimensionalen Raum $\R^d, (d \ll D)$ gleicht. 

Statt $K$ ist uns eine endliche Menge an $N$ Punkten $X=\seqx$, gegeben. 
$X$ wird als \textit{Punktwolke} bezeichnet und beschreibt in einem Experiment gemessene Daten, 
beispielsweise Sensormessdaten, biologische Informationen oder Bilddatens�tze. 

In Kapitel \ref{Experimente} werden wir uns mit einem 
Bilddatensatz mit \num{100000} Farbbildern mit einer Aufl�sung von $300 \times 300$ Pixeln 
besch�ftigen. Wir k�nnen diesen Datensatz als $N = \num{100000}$ elementige Punktwolke im 
$\R^{300 \times 300 \times 4}$ auffassen, wobei wir die vier Farbkan�le der Bilder ber�cksichtigen.

Wir gehen dabei davon aus, dass $K$ und $X$ in einem gewissen Sinne \enquote{�hnlich} sind. 
Nun ist es Ziel der TDA mittels Methoden der Topologie Aussagen �ber die Struktur von $X$ zu treffen, 
welche, aufgrund der �hnlichkeit, auch f�r $K$ gelten sollen. Die $K$ zugrundeliegende Struktur 
kann uns dann dabei helfen Aussagen �ber weitere gemessene Daten zu treffen, 
da diese auch in der uns unbekannten Menge $K$ liegen sollten. Zus�tzlich hilft
% TODO: erg�nzend siehe Oudot, Ch.4

%-----------------------------------
%----------------------------------------------------------------------------------------

% (Eher sp�ter in Kap2) Simpliziale Mengen sind einer Form um topologische R�ume kombinatoricsh darzustellen 
% deshalb eignen sie sich besonders f�r Berechnungen.

%----------------------------------------------------------------------------------------

\section{Dimensionsreduktion}

% hoch dimensionale Daten: Bilder, Word-count Vektoren, Zell Daten

% Ganz urspr�nglich Chernoff feces, dimensionen und knoten sie tSNE p.1

Sei $X=\seqx$ wir bezeichnen $D$ als die Dimension unserer Daten, beziehungsweise als die Anzahl 
der gemessenen Eigenschaften. $N$ ist die Anzahl der verf�gbaren Datenpunkte. In der Praxis k�nnen $D$ und $N$ 
sehr gro� sein. So gilt f�r den Bilddatensatz welchen wir in Kapitel \ref{Experimente} betrachten werden, 
$N=\num{100000}, D=\num{360000}$. %TODO: betrachten wir cartoonset100k oder cartoonset10k?

% Wahrscheinlichkeitstheorie:      |  Geometrie:
% Probabilistic PCA, VAEs, GANs    |  tSNE  (lt. Posada)

% Dann �berleiten zu UMAP

\textit{Hier sollen die zwei Arten an DR Algorithmen vorgestellt werden (Matrix Faktorisierung 
und Graph Layout). Zus�tzlich sollen PCA, Isomap, Laplacian Eigenmaps und t-SNE vorgestellt werden.} \\

Die meisten dimensionsreduktions Verfahren beruhen auf der Annahme, das reale hochdimensionale Daten 
sich in der Umgebung einer niedrigdimensionalen Mannigfaltigkeit konzentrieren. 
In der Literatur ist diese Annahme als Mannigfaltigkeits Hypothese (\textit{engl.: manifold hypothesis})
bekannt \cite{Mitter,Rifai}. 
Ans�tze um einen gegebenen Datensatz auf diese Annahme zu testen 
finden sich in \cite{Fefferman}.  % TODO: zu konkret?

% TODO: betone Gleichverteilung auf Mannigfaltigkeit bei UMAP

% TODO: warum ist Topologie sinnvoll?

% TODO: Betrachte Oudot s.67

% TODO: Anwendungen von persistence Oudot s.8

%----------------------------------------------------------------------------------------

\section{Eigene Beitr�ge}

Wir m�chten nun die Ziele dieser Arbeit formulieren.
\begin{itemize}
	\item Einf�hrung in die grundlegenden Werkzeuge der topologischen Datenanalyse
	\item M�gliche Schritte wie die L�cke zwischen Theorie und Praxis des UMAP Verfahrens 
		  geschlossen werden kann
	\item UMAP anhand sinnvoller Datens�tze mit anderen Dimensionsreduktionsverfahren vergleichen
	\item 
\end{itemize}
Insbesondere werden wir die in Kapitel 2 von \cite{UMAP} beschriebene Theorie ausf�hrlicher 
beschreiben und in einen allgemeineren Kontext setzen.

%----------------------------------------------------------------------------------------

\section{Gliederung}

In Kapitel \ref{Grundlagen} werden wir die theoretische Grundlage des UMAP Verfahrens erkl�ren. 
Dies wird einige Definitionen und Grundlagen aus der Kategorientheorie und der 
(Algebraischen-)Topologie erfordern. Wir haben mich bem�ht diese m�glichst vollst�ndig darzustellen.
F�r zus�tliche Informationen empfielt sich \cite{Brandenburg,Barr,Munkres}.

Kapitel \ref{UMAP} wird die Theorie des UMAP Verfahrens darstellen. Dabei werden wir 
auf die in \ref{Grundlagen} gelegten Grundlagen zur�ckgreifen. Zus�tzlich soll argumentiert 
werden, dass eine praktische Implementierung wie sie in der Theorie beschrieben ist zu 
rechenaufwendig ist und somit wenig praktischen nutzen hat. Deshalb 

%----------------------------------------------------------------------------------------

%!TEX root = ../main.tex

\chapter{Grundlagen} 

\label{Grundlagen}

%----------------------------------------------------------------------------------------

%Define math commands used just in this chapter
\renewcommand{\C}{\mathscr{C}}  % Symbol for a category C
\newcommand{\D}{\mathscr{D}}  % Symbol for a category D
\renewcommand{\N}{\mathbb{N}}  % Symbol for natural numbers
\newcommand{\M}{\mathcal{M}}  % Symbol for Manifold M
\newcommand{\cech}{\u{C}ech}  % Symbol for Cech complex
\newcommand{\Set}{\mathbf{Set}}  % Symbol for category of sets
\newcommand{\sSet}{\mathbf{sSet}}  % Symbol for category of sets
\newcommand{\Top}{\mathbf{Top}}  % Symbol for category of sets %TODO: uncomment
\renewcommand{\hom}{\mathsf{Hom}}  % Symbol for morphisms

\makeatletter
\newcommand{\colim@}[2]{%
  \vtop{\m@th\ialign{##\cr
    \hfil$#1\operator@font colim$\hfil\cr
    \noalign{\nointerlineskip\kern1.5\ex@}#2\cr
    \noalign{\nointerlineskip\kern-\ex@}\cr}}%
}
\newcommand{\colim}{\mathop{\mathpalette\colim@{\rightarrowfill@\textstyle}}\nmlimits@}
\makeatother

%----------------------------------------------------------------------------------------

% Struktur Posada:
% Kategorientheorie (Kategorien, Funktoren, nat�rliche Transformationen)
% Dieses allgemeine Ger�st l�sst einen leicht simpliziale Komplexe zu simplizialen Mengen
% veralgemeinern, unischere Mengen einf�hren und die f�r UMAP verwendete Theorie erl�utern.
% discrete, free and monoid category
% category Set, Top, duale Kategorie kurz (morphismen umkehren)

% Topologischer Raum und Homeomorphismen

% Simplex, simplizialer Komplex, abstrakter simplizialer Komplex

% Zusammenfassen der bisherigen Konstruktion und das man einen Schritt weiter geht
% Ausblicke auf das was man sp�ter mit einer Definition macht
% Wof�r war die gerade gemachte Definition sinnvoll
% Die Definition einer Adjunktion ist daf�r wichtig einen zwischen metrischen R�umen und 
% unsicheren simplizialen Mengen zu �bersetzen.
% Typische Verwendungen f�r die Definition
 
%----------------------------------------------------------------------------------------

% TEXTBAUSTEINE

% Adjoint functor theorems are theorems stating that under certain conditions a functor that preserves limits is a right adjoint, and that a functor that preserves colimits is a left adjoint.

% In der Einleitung bereits erw�hnt nehmen wir an, dass die zu untersuchenden Daten $X \subset \R^D$
% auf einer $d$-dimensionalen Mannigfaltigkeit $(d \ll D)$ liegen. Ein passendes $d$ zu finden ist nicht offensichtlich.
% Wir verweisen den Leser auf... %TODO: Quellen f�r Sch�tzung abgeben.

%----------------------------------------------------------------------------------------
%----------------------------------------------------------------------------------------

% \section{Einleitung}
	Um die Geometrie \enquote{mathematischer R�ume} zu beschreiben, ist es aus topologischer Sicht oft ausreichend,
	die \enquote{L�cher} der R�ume zu charakterisieren. 
	Um diese Aussage zu formalisieren, und damit die Grundlage f�r das UMAP Verfahren zu legen, 
	werden wir einige Begriffe aus der (algebraischen) Topologie ben�tigen. 

	Die grundlegenden Definitionen \textit{topologischer und metrischer R�ume} werden uns helfen 
	\textit{(riemannsche) Mannigfaltigkeiten} einzuf�hren. Der Begriff der Mannigfaltigkeit formalisiert den 
	niedrigdimensionalen Raum auf welchem unsere Daten liegen.

	Die geometrische Struktur dieser R�ume werden wir mit Hilfe der \textit{Homologie} einf�hren
	und sehen wie diese visuelle Eigenschaften beschreibt. Um diese strukturelle Darstellung auf eine endliche
	Punktwolke anwenden zu k�nnen, werden wir den Begriff der \textit{persistenten Homologie} einf�hren.

	Die Begriffe und ben�tigten Aussagen werden vollst�ndig und anschaulich eingef�hrt. 
	An einigen Stellen werden wir den Leser auf geeignete erg�nzende Literatur verweisen. 

%----------------------------------------------------------------------------------------

\section{Topologische R�ume}
	Der Grundlegende Begriff der Topologie ist der des topologischen Raumes.

	% TODO: Definition topologischer Raum
	\begin{defn}[Topologischer Raum]
		Eine Topologie ist ein Mengensystem $T$ ... 
	\end{defn}
	% TODO: Brauchen wir Beispiele?

	Man kann den Begriff erweitern indem man einen Abstandsbegriff auf der Menge $X$ definiert. Dies f�hrt uns 
	zum metrischen Raum.

	% TODO: Definition metrischer Raum
	\begin{defn}[Metrischer Raum]
		Eine Metrik ist eine Abbildung,...
	\end{defn}
	% TODO: Brauchen wir Beispiele?

	In der Einleitung bereits erw�hnt, werden wir Annehmen, dass unsere Daten $X=\{x_i\}_{i=1}^N, (x_i \in \R^D)$ mit $D$ 
	gemessenen Eigenschaften mittels $d, (d \ll D)$ Eigenschaften dargestellt werden k�nnen. Dies werden 
	wir formalisieren indem wir Mannigfaltigkeiten einf�hren. Anschaulich ist eine $d$-dimensionale Mannigfaltigkeit
	ein topologischer Raum welcher lokal dem euklidischen Raum $\R^d$ gleicht.

	% TODO: Definition Mannigfaltigkeit
	\begin{defn}[Mannigfaltigkeit]
		Sei $\M$ ein topologischer Raum,...
	\end{defn}

	\begin{rem}
		Eine \textit{differenzierbare} Mannigfaltigkeit ist eine...
	\end{rem}

	\begin{ex}
		$S^n$ ist eine $n$-dimensionale Mannigfaltigkeit im $\R^{n+1}$.
	\end{ex}

	Die Annahme, dass wir nur $d$ Eigenschaften ben�tigen um unsere Daten $X$ darzustellen, l�sst sich formell 
	ausdr�cken, dass $X$ eine $d$-dimensionale Mannigfaltigkeit im $\R^D$ ist.

	Wir k�nnen eine Mannigfaltigkeit $\M$ mit einem Abstandbegriff erweitern. Diese zus�tzliche 
	Regularit�tseigenschaft erm�glicht es uns von Distanzen, Winkeln und Kurven auf $\M$ zu sprechen. 
	Diesen Abstandsbegriff nennen wir \textit{riemannsche Metrik} und er ist wie folgt definiert:

	% TODO: Definition riemannsche Mannigfaltigkeit
	\begin{defn}[Riemannsche Mannigfaltigkeit]
		Sei $g$,...
		Ein Tupel $(\M, g)$, wobei $\M$ eine Mannigfaltigkeit und $g$ eine riemannsche Metrik ist hei�t
		\textit{riemannsche Mannigfaltigkeit}.
	\end{defn} 

	% TODO: Metrisierbarkeit riemannsche Mannigfaltigkeit
	\begin{rem}
		Eine riemannsche Mannigfaltigkeit ist stets metrisierbar im Sinne, das ...
	\end{rem}

	% TODO: Beispiele riemannsche Mannigfaltigkeit

	Die Riemannmetrik beschreibt eine Distanz auf der Mannigfaltigkeit. Die L�nge eines k�rzesten Weges auf $\M$ 
	zwischen zwei Punkten $p, q \in \M$ wird als Geod�te bezeichnet und ist definiert als 

	\begin{defn}[Geod�te]
		Seien $p, q \in \M$ ...
	\end{defn}

	Der Begriff der riemannschen Mannigfaltigkeit erm�glicht es uns unsere zentrale Annahme, das $X \subseteq \R^D$ 
	einer niedrigdimensionalen Struktur entnommen ist, zu formalisieren.
	Wir m�chten nun diese niedrigdimensionale Struktur genauer beschreiben.

%----------------------------------------------------------------------------------------

% TODO: Die Aussagen aus diesem Abschnitt sollen in Appendix gegeben werden (evtl Cech und VR unten geben.)
\section{Homologie}
	% Hom�omorphismus, Homologie (Gruppen), homologie Equivalenz,
	Eine bekannte Aussage, welche der Topologie entstammt, ist, dass eine Kaffeetasse das Gleiche \textit{topologische Objekt} 
	wie ein Donut ist. Dies ist dadurch begr�ndet, dass man eine Kaffeetasse durch strecken und stauchen 
	(ohne rei�en) in einen Donut transformieren kann. Diese \textit{Gleichheit} l�sst sich wie folgt mathematisch darstellen:

	\begin{defn}[Stetige Abbildung]
		Sei ...
	\end{defn}

	\begin{defn}[Hom�omorphismus]
		Sei ...
	\end{defn}

	Nun werden wir den Begriff der Homologie einf�hren. Daf�r werden wir den Begriff der simplizialen Komplexe 
	ben�tigen um zuerst die simpliziale Homologie einzuf�hren.

	\begin{defn}[Simplizialer Komplex]
		Sei ...
	\end{defn}

	\begin{defn}[Abstrakter simplizialer Komplex]
		Sei ...
	\end{defn}

	\begin{ex}
		\cech-Komplex
	\end{ex}

	\begin{ex}
		VR-Komplex
	\end{ex}

	\begin{defn}[Homologie]
		Sei ...
	\end{defn}

	\begin{defn}[Homologiegruppe]
		Sei ...
	\end{defn}

	\begin{ex}
		Die Homologiegruppen eines Balls, Donuts, Graphen, ...
	\end{ex}

	\begin{defn}[Filtration]
		Sei $T \subseteq \R$, eine \textit{Filtration} $\mathcal{X}$ �ber $T$ ist eine Familie von 
		topologischen R�umen $\{X_i\}_{i \in T}$, so dass $X_i \subseteq X_j$ 
		f�r $i \leq j \in T$.  % (Oudot s.29)
	\end{defn}

	\begin{rem}
		Eine Filtration $\mathcal{X}$ wird meist nicht als Familie verschiedener topologischer 
		R�ume $X_i$ angesehen, sondern als ein einziger Raum, welcher sich im Laufe der Zeit 
		\enquote{transformiert}. Siehe dazu \cite{Oudot}.  % (Oudot s.30)
		Dies stimmt mit unserem Bild �berein, den $\epsilon$-Parameter eines \cech-Komplexes %TODO: Satz l�schen und motivation cech komplex nennen
		immer gr��er werden zu lassen. Wir werden \textit{persistente Homologie} nutzen um 
		die homologischen Eigenschaften zu beschreiben, welche f�r alle $i \in T$ erhalten 
		(bzw. persistent) bleiben.
	\end{rem}

	% TODO: Oudot Kap. 4,5 f�r Anwendungen von persistenter Homologie

	% TODO: Bemerkung, dass \mathcal{X} eine Repr�sentation eines Poset ist (siehe Oudot s.29)

	\begin{ex}
		% TODO: warum?
		Insbesondere ist somit der \cech-Komplex eine Filtration, da ...
	\end{ex}

	\begin{ex}
		% TODO: warum?
		Eine weitere Filtration ist der Vietoris-Rips-Komplex
	\end{ex}

	\begin{rem} % TODO: Bem in Def umwandeln
		Der VR-Komplex ist ein Beispiel eines Clique-Komplexes und wird als solcher 
		eindeutig durch sein 1-Skelet identifiziert. Sei $V$ ein VR-Komplex und $V^1$ das 
		zugeh�rige 1-Skelet. Dann erh�lt man $V$ aus $V^1$, indem man alle Cliquen im 
		$V^1$ zugrundeliegenden Graphen bestimmt. Eine k-Clique ist ein vollst�ndiger Subgraph 
		mit k Knoten. 
	\end{rem}

	Es gibt weitere Formen von Komplexen, die bekanntesten sind die CW-Komplexe und Zellkomplexe. 
	Ziel ist es stets mit Hilfe einfacher Konstrukte die Topologie eines Raums zu beschreiben.

	Die vom \cech-Komplex und vom Vietoris-Rips-Komplex beschriebene Homologie kann sich unterscheiden (siehe Abbildung \ref{fig:Cech_VR}).

	\begin{figure}
		%\centering
		\includegraphics[width=400px, height=274px]{Figures/Cech_VR_complex}  % width=1\textwidth
		%\decoRule
		\captionsetup{justification=RaggedRight}  % centerfirst
		\caption[Versch. Komplexe]{Eine Punktwolke [oben links] kann zu einem \cech-Komplex [unten links] 
								   oder einem VR-Komplex [unten rechts] basierend auf dem Parameter $\epsilon$ 
								   konstruiert werden. Die Homotopietypen der $\epsilon/2$ �berdeckung vom 
								   \cech-Komplex ($S^1 \vee S^1 \vee S^1$), w�hrend das VR-Komplex die Homotopietypen 
								   ($S^1 \vee S^2$) besitzt. (\textit{Quelle:} \cite{Barcodes})}

		\label{fig:Cech_VR}
	\end{figure}

	Wir ben�tigen nun noch den Begriff einer \textit{guten �berdeckung} um dann eine geeignete Aussage �ber einen \cech-Komplex einer Menge zu machen. 

	\begin{defn}[�berdeckung]
		Eine �berdeckung eines topologischen Raumes ist \dots
		Eine gute �berdeckung ist \dots
	\end{defn}

	\begin{NT}  % Nerv Theorem
		% Ein top. Raum ist homotopie equivalent zu einer endlichen guten �berdeckung. 
		Sei ...
		\label{thm:NT}
	\end{NT}
	
	Das Nerv Theorem \ref{thm:NT} liefert uns eine Aussage dar�ber, dass ein topologischer Raum $X$ zu einer endlichen guten 
	�berdeckung von $X$ homotopie�quivalent ist.
	Wie wir sehen ist die Konstruktion stark abh�ngig vom $\epsilon$-Parameter. Um dem 
	entgegenzuwirken nutzt man die Idee der \textit{persistenten Homologie}. Dabei 
	wird betrachtet wie sich die Homologie eines Raumes f�r eine monoton wachsende Folge 
	$(\epsilon_i)_{i \in \N}$ verh�lt.

	F�r den von uns betrachteten Fall, die Struktur einer Riemannschen Mannigfaltigkeit $(\M, g)$ zu beschreiben, werden wir 
	Annehmen, dass die Daten $\mathbf{x}_i \in X$ gleichverteilt bez�glich der Riemannmetrik $g$ sind, bzw. $g$ so w�hlen, das 
	die Annahme erf�llt ist. Somit erhalten wir eine gute �berdeckung ohne eine Abh�ngigkeit von $\epsilon$ zu haben. 

%----------------------------------------------------------------------------------------

\section{Kategorientheorie}
	Die f�r die mathematischen Grundlagen des UMAP Verfahren ben�tigten Definitionen 
	werde ich mithilfe der Kategorientheorie einf�hren. 
	Diese sehr abstrakte Form mathematische Objekte und Zusammenh�nge zu formalisieren 
	wurde erstmals in den vierziger Jahren von Samuel Eilenberg und Saunders Mac Lane eingef�hrt. \\
	Die Definitionen sind dem Buch von Brandenberg \cite{Brandenburg} entnommen. 
	 
	\begin{defn}[Kategorie]
	 Eine Kategorie $\C$ besteht aus folgenden Daten:
	 	\begin{enumerate}
	 		\item Eine Klasse $Ob(\C)$, deren Elemente wir \textit{Objekte} nennen
			\item zu je zwei Objekten $\mathit{A, B \in Ob(\C)}$ einer Menge $\mathit{\hom_\C(A, B)}$, 
				deren Elemente wir mit $f: A \to B$ notieren und \textit{Morphismen} von $A$ nach $B$
				nennen,
			\item zu je drei Objekten $A, B, C \in Ob(\C)$ einer Abbildung
				\begin{equation*}
					\hom_\C(A, B) \times \hom_\C(B, C) \to \hom_\C(A, C)
				\end{equation*}
				die wir mit $(f, g) \mapsto g \circ f$ notieren und \textit{Komposition von Morphismen} nennen,
			\item zu jedem Objekt $A \in Ob(\C)$ einen ausgezeichneten Morphismus
				\begin{equation*}
					id_A \in \hom_\mathcal{C}(A,A),
				\end{equation*}
				welchen wir die \textit{Identit�t} von $A$ nennen.
		\end{enumerate}
	 Diese Daten m�ssen den folgenden Regeln gen�gen:
		\begin{enumerate}
			\item Die Komposition von Morphismen ist \textit{assoziativ}: F�r drei Morphismen 
				der Form $f:A \to B, \ g:  B \to C, \ h:C \to D$ in $\C$ gilt
				\begin{equation*}
					h \circ (g \circ f) = (h \circ g) \circ f
				\end{equation*}
				als Morphismen $A \to D$.
			\item Die Identit�t sind \textit{beidseitig neutral} bez�glich der Komposition:
				F�r jeden Morphismus $f: A \to B$ in $\C$ gilt
				\begin{equation*}
					f \circ id_A =  f = id_A \circ f
				\end{equation*}
		\end{enumerate}
	\end{defn}

	\begin{rem}
		Anstelle von $A \in Ob(\C)$ schreibt man meistens $A \in \C$. 
		Falls die Kategorie $\C$ aus dem Kontext bekannt ist, werden wir $\hom_\C(A, B)$
		mit $\hom(A, B)$ abk�rzen.
	\end{rem}

	\begin{rem}
		% TODO: Besser formulieren
		Eine Klasse ist eine Menge, welche zu gro� ist um eine Menge zu sein. 
		F�r eine Definition einer Klasse verweisen wir auf \cite{Levy}. 
		In unseren Beispielen gen�gt die Vorstellung einer Menge.
		Meist ist $Ob(\C)$ sogar eine Menge. Dann spricht man formal von einer strikten kleinen Kategorie.
	\end{rem}

	\begin{ex}
		\textit{Passende Beispiele von Kategorien, welche sp�ter wieder genutzt werden. $\Top, \Set$}
	\end{ex}

	Ein weiterer f�r die folgenden Definitionen wichtiger Begriff ist der der dualen Kategorie.

	\begin{defn}[Duale Kategorie]
		Es sei $\C$ eine Kategorie. Dann k�nnen wir eine neue Kategorie $\C^{op}$	 konstruieren: 
		Sie besitzt dieselben Objekte wie $\C$, allerdings werden die Morphismen \enquote{umgedreht}:
		F�r $A,B \in \C$ sei
		\begin{equation*}
			\hom_{\C^{op}}(A, B) \coloneqq \hom_\C(B, A).
		\end{equation*}
		Die Identit�ten ver�ndern sich nicht. Die Komposition
		\begin{equation*}
			\circ^{op} : \hom_{\C^{op}}(A, B) \times \hom_{\C^{op}}(B, C) \to \hom_{\C^{op}}(A, C) 
		\end{equation*}
		ist durch
		\begin{equation*}
			\hom_\C(B, A) \times \hom_\C(C, B) \cong \hom_\C(C, B) \times \hom_\C(C, B) \times \hom_\C(B, A) \mathring{\longrightarrow} \hom_\C(C, A)
		\end{equation*}
		definiert, d.h. $f \circ^{op} g \coloneqq g \circ f$. Auf diese Weise ist  $\C^{op}$ tats�chlich 
		eine Kategorie und hei�t die \textit{zu} $\C$ \textit{duale Kategorie}.
	\end{defn}

	\begin{rem}
		Eine Eigenschaft der dualen Kategorie ist, dass Aussagen, 
		welche f�r alle Kategorien bewiesen wurden, auch f�r alle dualen Kategorien gelten. 
	\end{rem}

	Wir m�chten nun den Begriff des Funktors zwischen zwei Kategorien einf�hren. Ein Funktor ordnet Objekte einer 
	Kategorie $\C$ Objekten einer Kategorie $\D$ zu, und entsprechend f�r Morphismen. Insbesondere bleibt 
	die Eigenschaft der Isomorphie zwischen zwei Objekten erhalten. 
	%TODO: Was ist ein isomorpher Funktor?

	\begin{defn}[Funktor]
		Es seien $\C$ und $\D$ zwei Kategorien. Ein \textit{Funktor}
		\begin{equation*}
			F : \C \to \D
		\end{equation*}
		von $\C$ nach $\D$ besteht aus folgenden Daten:
		\begin{enumerate}
			\item f�r jedes Objekt $A \in \C$ ein Objekt $F(A) \in \D$,
			\item f�r jeden Morphismus $f: A \to B$ in $\C$ einen Morphismus
						\begin{equation*}
							F(f) : F(A) \to F(B)
						\end{equation*}
						in $\D$.
			\end{enumerate}
		Dabei soll gelten:
		\begin{enumerate}
			\item F�r jedes Objekt $A \in \C$ ist $F(id_A) = id_{F(A)}$.
			\item F�r je zwei Morphismen $f: A \to B,\ g: B \to C$ in $\C$ gilt in $\D$:
						\begin{equation*}
							F(g \circ_{\C} f) = F(g) \circ_{\D} F(f)
						\end{equation*}
		\end{enumerate}
	\end{defn}

	\begin{rem}
		Bez�glich der Kategorie $\C$ ist ein Funktor $F : \C \to \D$ kovariant, w�hrend 
		$F : \C^{op} \to \D$ kontravariant (bzgl. $\C$) ist.
	\end{rem}

	\begin{rem}
		Insbesondere kann man f�r eine Kategorie $\C$ und Objekte $A, B, C \in \C$ den \textit{Hom-Funktor} 
		definieren, indem man 

		\begin{equation}
			\label{Hom-Funktor}
			\hom(-,B): \C \to \Set
		\end{equation}

		betrachtet. Der Hom-Funktor bildet ein Objekt $A \in \C$ auf die Menge der Morphismen $\hom(A, B)$ ab, 
		und einen Morphismus $h: A \to C$ auf die Funktion 

		\begin{equation}
			\hom(h, B): \hom(C, B) \to \hom(A, B), \text{wobei } g \mapsto g \circ h \text{ f�r } g \in \hom(C, B)	
		\end{equation}

	\end{rem}

	Eine h�ufig verwendete Form eines kontravarianten Funktors ist die Pr�garbe 
	(\textit{engl.: presheaf}). Wir werden diesen Funktor sp�ter verwenden um 
	\textit{simpliziale Mengen} einzuf�hren. % TODO: weglassen oder genauer beschreiben?

	\begin{defn}[Pr�garbe]
		Eine Pr�garbe auf einer kleinen Kategorie $\C$ ist ein Funktor
		\begin{equation*}
			F: \C^{op} \to \Set
		\end{equation*}
		von der dualen Kategorie $\C^{op}$ von $\C$ in die Kategorie $\Set$ von Mengen.
	\end{defn}

	\begin{defn}[Pr�garbenkategorie]
		Sei $\widehat{\C}$ die Pr�garbenkategorie einer Kategorie $\C$ : Objekte sind die 
		Funktoren $F: \C^{op} \to \Set$, und Morphismen sind nat�rliche 
		Transformationen der Funktoren.
	\end{defn}

	% TODO: wird dies ben�tigt?
	\begin{rem}
		Allgemeiner kann man auch die Kategorie $[\C, \D]$ einf�hren, deren Objekte 
		Funktoren $F: \C \to \D$ sind und deren Morphismen ebenfalls nat�rliche 
		Transformationen der Funktoren sind.
	\end{rem}


	% TODO: Definiere nat�rliche Transformationen. 
	% https://ncatlab.org/nlab/show/natural+transformation

	\begin{YL}
		Random Text
		% TODO: add Yoneda lemma
	\end{YL}

	\begin{defn}[Adjunktion]
		% TODO: f�ge Definition einer Adjunktion ein
	\end{defn}

%----------------------------------------------------------------------------------------

\section{Simpliziale Mengen}
	F�r die Konstruktion der topologischen Repr�sentation der Daten werden wir \textit{simpliziale Mengen} 
	ben�tigen. Diese stellen eine Verallgemeinerung der in der TDA h�ufig verwendeten \textit{Simplizialkomplexe} 
	dar. Wir m�chten den interessierten Leser an dieser Stelle auf die sehr verst�ndlich und ausf�hrlich gestalteten 
	Notizen von Friedman \cite{Friedman} verweisen, dort wird der Unterschied zwischen diesen beiden Konstrukten 
	sehr illustrativ erl�utert. % TODO: Appendix mit Zusammenfassung? 

	Um simpliziale Mengen einzuf�hren werden wir die kategorientheoretische Definition verwenden. 
	Daf�r wird die \textit{Simplexkategorie} ben�tigt. Zuerst werden wir der Vollst�ndigkeit halber die Definition eines 
	\textit{geometrischen Simplex} geben.

	\begin{defn} %[(geometrischer) Simplex]
		Ein \textit{(geometrischer) $n$-Simplex} ist eine von $n+1$ geometrisch unabh�ngigen Punkten $\{v_0,\dotsc,v_n\}$ 
		aufgespannte konvexe H�lle im euklidischen Raum. Die geometrische Unabh�ngigkeit der Vektoren bedeutet dabei, das 
		$v_1 - v_0, \dotsc v_n - v_0$ linear unabh�ngig sind. 
		Die Vektoren $v_i$ werden \textit{Knoten} genannt und die Teilmengen von $\{v_0,\dotsc,v_n\}$, 
		\textit{Facetten} des $n$-Simplex.
	\end{defn}

	\begin{defn}[Simplexkategorie]
		Die Objekte der \textit{Simplexkategorie} $\Delta$ sind die Mengen 
		$[n] \coloneqq \{0,1,\dotsc,n\}$ f�r $n \in \mathbb{N}$, 
		und Morphismen sind monoton wachsende Abbildungen. 
	\end{defn}

	\begin{figure}
		%\centering
		\includegraphics[width=400px, height=146px]{Figures/geom_simplizes}
		%\decoRule
		\caption[Geometrische Simplizes]{Die geometrischen 0-, 1-, 2- und 3-Simplizes, 
				in manchen F�llen werden diese auch als (geordnete) Standardsimplizes bezeichnet.}
		\label{fig:geom_simplizes}
	\end{figure}

	Um eine Verbindung zwischen der Simplexkategorie und geometrischen $n$-Simplizes (siehe Abbildung \ref{fig:geom_simplizes}) herzustellen, 
	betrachtet man die $n$-elementigen Mengen in $\Delta$ und den Funktor $\abs{\cdot}: \Delta \to \Top$, 
	gegeben durch

	\begin{equation*}
		\abs{\cdot} : [n] \mapsto \abs{\Delta^n} \coloneqq \biggl\lbrace (t_0, \dotsc, t_n) \in \R^{n+1} 
		\Big\vert \sum_{i=0}^n t_i = 1, t_i \geq 0 \biggr\rbrace
	\end{equation*}

	Das Bild von $[n]$ unter $\abs{\cdot}$ ist somit ein geometrischer $n$-Simplex.

	\begin{defn}
		Eine \textit{simpliziale Menge} ist ein Funktor 
		$X : \Delta^{op} \to \Set$. �blicherweise wird 
		$X([n])$ als $X_n$ geschrieben, und wir bezeichnen die Elemente $x \in X_n$ als n-Simplizes. 
		Der n-dimensionale \textit{Standardsimplex} ist 
		
		\begin{equation*}
			\Delta^n \coloneqq \hom(-, [n]).
		\end{equation*}
	\end{defn}

	\begin{rem}
		Simpliziale Mengen sind also Hom-Funktoren (siehe Gleichung (\ref{Hom-Funktor})).
	\end{rem}

	\begin{defn}
		Die Objekte der Kategorie der simplizialen Mengen $\sSet$ sind simpliziale Mengen und ihre Morphismen sind nat�rliche Transformationen.
	\end{defn}

	Um ein besseres Verst�ndnis f�r eine simpliziale Menge zu bekommen ist es oft sinnvoll an einen \textit{Simplizialkomplex} 
	zu denken. 

	\begin{defn} %[Simplizialkomplex]
		Ein \textit{(geometrischer) Simplizialkomplex} $\mathcal{K}$ in $\R^n$ ist eine Menge von (geometrischen) Simplizes, 
		m�glicherweise unterschiedlicher Dimension, in $\R^n$, so dass
		\begin{enumerate}
			\item jede Facette eines Simplex aus $\mathcal{K}$ in $\mathcal{K}$ ist, und
			\item der Schnitt zweier Simplizes aus $\mathcal{K}$ ist eine Facette beider Simplizes.
		\end{enumerate}
	\end{defn}

	\begin{rem}
		Ein Simplizialkomplex ist anschaulich betrachtet ein geometrisches Objekt, welches aus mehreren Simplizes \textit{zusammengef�gt} wird. 
		Die Simplizes d�rfen dabei nur entlang ihrer Facetten \textit{zusammengef�gt} werden. 
	\end{rem}

	Ein Simplizialkomplex kann dabei helfen einen topologischen Raum zu beschreiben, die kombinatorische Struktur des 
	Simplizialkomplexes kann dann dazu genutzt werden Aussagen �ber den zugrundeliegenden topologischen Raum zu treffen. 
	Dabei sind die genauen r�umlichen Lagebeziehungen der Simplizes oft zu vernachl�ssigen und es kann folgende Verallgemeinerung gemacht werden:  

	\begin{defn} %[Abstrakter Simplizialkomplex]
		Ein \textit{abstrakter Simplizialkomplex} besteht aus einer Menge $S^0$ an \textit{Knoten} und, f�r jedes $k$, einer Menge $S^k$, bestehend aus 
		Teilmengen von $S^0$ der Kardinalit�t $k+1$, so dass jede $(i+1)$-elementige Teilmenge einer Menge aus $S^k$ auch in $S^i$ ist.
		Die Menge $S^k$ wird \textit{$k$-Skelett} genannt.
	\end{defn}

	\begin{rem}
		Analog zum $k$-Skelett eines Simplizialkomplexes werden wir im folgenden die $k$-Simplizes einer simplizialen Menge als $k$-Skelett bezeichnen. 
		F�r simpliziale Mengen wird diese Bezeichnung nicht konsistent in der Literatur genutzt. 
	\end{rem}

	Abstrakte Simplizialkomplexe besitzen im Allgemeinen also keine Informationen �ber die relative r�umliche Lage der Knoten, insbesondere k�nnen die 
	Knoten beliebige Objekte sein. �hnlich sind simpliziale Mengen zu verstehen, allerdings enthalten diese noch \textit{mehr} Informationen �ber die 
	Simplizes, siehe dazu \cite{Friedman}. 

	Eine hilfreiche Eigenschaft (abstrakter) Simplizialkomplexe ist, dass sich diese aus den einfach zu beschreibenden geometrischen $n$-Simplizes zusammensetzen. 
	Diese Eigenschaft l�sst sich auf simpliziale Mengen mittels Yoneda Lemma �bertragen. Sei $X$ eine simpliziale Menge, dann gibt es f�r alle $x \in X_n$ 
	einen Morphismus $x:\Delta^n \to X$.  Eine Anwendung von \cite{maclane} (�7, Thm. 1) liefert uns:  %TODO: pr�fe Quelle

	\begin{equation}
		X \simeq \varinjlim \Delta^n,
		\label{eq:colimit}
	\end{equation}

	wobei der Kolimit �ber eine von $X$ bestimmte Indexkategorie genommen wird.

	\begin{rem}
		Eine mathematische Definition des Kolimes wird in dieser Arbeit nicht gegeben, da diese einige Vorbereitungen ben�tigen w�rde. Wir verweisen den 
		Leser auf geeignete Literatur, beispielsweise \cite{Brandenburg}.
	\end{rem}

	Dennoch soll kurz erl�utert werden wie der Kolimes zu verstehen ist. Er ist das duale Konstrukt zum Limes, deshalb operiert er auf der dualen Kategorie. 
	Anschaulich werden die Objekte (hier die $\Delta^n$) in einer passenden Weise \textit{zusammengef�gt}. 
	In Gleichung (\ref{eq:colimit}) bedeutet dies also, dass sich eine simpliziale Menge aus den Standardsimplizes zusammensetzt.


	Wie bereits erw�hnt lassen sich f�r topologische R�ume geeignete (abstrakte) Simplizialkomplexe konstruieren, �hnliches gilt auch f�r simpliziale Mengen. 
	In der Tat gibt es aus kategorientheoretischer Sichtweise eine \textit{gute} Beziehung zwischen simplizialen Mengen und topologischen R�umen, diese l�sst 
	sich durch zwei adjungierte Funktoren wie folgt beschreiben: 

	\begin{thm}
		Die \textit{geometrische Realisierung} gegeben durch:
		
		\begin{equation}
			\quad \abs{\cdot} : \sSet \to \Top,		\quad \abs{X} \mapsto \varinjlim \abs{\Delta^n},
		\end{equation}

		und der \textit{singul�re Mengen} Funktor

		\begin{equation}
			\quad S : \Top \to \sSet,		\quad S(Y) \mapsto \hom_\Top(\abs{\Delta^n}, Y),
		\end{equation}

		bilden eine Adjunktion.
	\end{thm}

	Auf diese Adjunktion werden wir in Kapitel \ref{UMAP} zur�ckkommen und diese f�r metrische R�ume anpassen. 

%----------------------------------------------------------------------------------------

\section{Unscharfe Mengen}
	Ein geometrischer Simplizialkomplex enth�lt Informationen �ber die Lage der Knoten, abstrakte Simplizialkomplexe und simpliziale Mengen 
	fehlt diese Eigenschaft. In Kapitel \ref{UMAP} werden wir die hochdimensionalen Daten $\mathbf{x}_i$ betrachten und diese indirekt 
	eine simpliziale Menge konstruieren. Dazu m�chten wir auch die Eigenschaft nutzen, dass f�r die Daten eine Metrik gegeben ist. 
	Um simplizialen Mengen, genauer gesagt den Knoten, einen \textit{Abstandsbegriff} zuzuordnen werden wir den Begriff der \textit{unscharfen Menge} nutzen.

	In der klassischen Mengentheorie ist die Zugeh�rigkeit eines Elementes $x$ zu einer Menge $X$ eine bin�re Funktion. Entweder gilt $x \in X \text{oder} x \notin X$. 
	Eine \textit{unscharfe Menge} verallgemeinert die Zugeh�rigkeit.

	\begin{defn}
		
	\end{defn}




%----------------------------------------------------------------------------------------




 
%!TEX root = ../main.tex

\chapter{UMAP} \label{UMAP} 

%----------------------------------------------------------------------------------------

% TODO: Wirkung des Funktors beschreiben \cite{Posada}

% Zeitplan:
% Sa: Kapitel 2,3,4 vervollst�ndigen, Einleitung und Appendix schreiben
% So: Plots plots plots, �berarbeiten 

%----------------------------------------------------------------------------------------
%----------------------------------------------------------------------------------------

% \section{Einleitung}
	In diesem Kapitel soll das UMAP Verfahren eingef�hrt werden. Dabei wird angenommen, 
	das die Daten $X=\seqx$ auf einer $d$-dimensionalen riemannschen Mannigfaltigkeit liegen. 

	Das UMAP Verfahren approximiert lokal die geod�tische Distanz der $\mathbf{x}_i$. 
	Dies f�hrt dazu, dass wir f�r jeden Datenpunkt $\mathbf{x}_i$ einen metrischen Raum $X_i$ erhalten. 
	Diese Konstruktion wird in Abschnitt \ref{sec:manifold} beschrieben. 

	Da die Metriken der $X_i$ a priori nicht miteinander kompatibel sind, wird in Abschnitt \ref{sec:repr} 
	die Adjunktion aus Satz \ref{thm:adj} auf metrische R�ume und unscharfe simpliziale Mengen erweitert. 
	Diese wird dazu genutzt die $X_i$ als unscharfe simpliziale Mengen darzustellen. Vereinigen wir die Mengen, 
	erhalten wir eine topologische Darstellung der hochdimensionalen Daten. Aufgrund der konstruierten Metriken 
	enth�lt diese lokale und aufgrund der unscharfen simplizialen Mengen globale Eigenschaften der Daten. 

	Um die Daten in den $\R^d$ einzubetten und somit zu einer niedrigdimensionalen Darstellung $Y$ 
	zu gelangen, wird in Abschnitt \ref{sec:einb} ebenfalls eine topologische Repr�sentation vom $\R^d$ 
	konstruiert. Die Angabe einer Funktion, welche den Unterschied der beiden Repr�sentationen darstellt, 
	erm�glicht uns dann die Repr�sentation vom $\R^d$ so zu optimieren, dass sie der Repr�sentation von $X$ 
	m�glichst �hnlich ist, somit erhalten wir eine $d$-dimensionale Darstellung $Y$ der Daten, welche mittels 
	eines geeigneten Funktors in einen metrischen Raum �berf�hrt werden kann. 

	Wir werden uns in diesem Kapitel nach der in McInnes et. al. \cite{UMAP} gew�hlten Beschreibung des UMAP 
	Verfahrens richten und diese insbesondere durch intuitive Erkl�rungen erg�nzen. 

%----------------------------------------------------------------------------------------

\section{Approximation der Mannigfaltigkeit} \label{sec:manifold}
	Wir nehmen nun an, dass $(\M,g)$ die $d$-dimensionale riemannsche Mannigfaltigkeit ist, auf welcher unsere Daten $X$ 
	liegen, also $X \subseteq \M$. 
	F�r den Fall, dass die Mannigfaltigkeit nicht bekannt ist, m�chten wir nun die Geod�ten auf $\M$, 
	und damit zwischen je zwei Datenpunkten auf $X$, approximieren. 
	Dazu nutzen wir folgendes Lemma:

	\begin{lem}
		Sei $p \in \M$ ein Punkt. Wenn 

		\begin{enumerate}
			\item $g$ lokal konstant auf einer offenen Umgebung $U$ von $p$ ist, 
				  so dass $g$ eine Diagonalmatrix bez�glich der Umgebungskoordinaten ist,
			\item $B_{r}(p) \subseteq U$ ein Ball mit Volumen $\frac{\pi^{\frac{n}{2}}}{\Gamma(\frac{n}{2}+1)}$ bez�glich $g$ ist,
		\end{enumerate}

		dann ist die Geod�te von $p$ zu jedem $q$ aus $B_{r}(p)$ durch $\frac{1}{r}d_{\R^n}(p,q)$ gegeben. 
		Dabei ist $d_{\R^n}$ die Metrik des Umgebungsraumes von $\M$ und $r$ der Radius von $B$ bez�glich des Umgebungsraumes.
		\label{lem:geod}
	\end{lem}

	\begin{rem}
		Ein Beweis des Lemmas findet sich in \cite{UMAP}. Die Idee l�sst sich wie folgt skizzieren. Die Determinante von $g$ 
		kann explizit angegeben werden, da das Volumen des Balls gegeben ist. Da $g$ zus�tzlich eine Diagonalmatrix ist l�sst sich $g$ 
		in diesem Fall eindeutig aus der Determinante bestimmen. Die explizite Form von $g$ erm�glicht es uns die Geod�te zwischen 
		$p$ und $q$ berechnen.
	\end{rem}

	Wir m�chten nun argumentieren, dass die beiden Bedingungen aus Lemma \ref{lem:geod} f�r unsere Daten erf�llt sind. 
	Die erste Bedingung ist erf�llt, falls wir annehmen, dass die Datenpunkte $\mathbf{x}_i$ gleichverteilt bez�glich $g$ auf $\M$ liegen. 
	Betrachten wir einen Ball $B_r$ auf $(\M,g)$, wobei $r$ so gew�hlt ist, dass $B_r$ genau $k, (k \in \N)$ Elemente aus $X$ enth�lt. 
	Da die $\mathbf{x}_i$ gleichverteilt bez�glich $g$ sind liegen in jedem $B'_r$ ebenfalls $k$ Elemente aus $X$. 
	Ein Ball $B(\mathbf{x}_i)$ welcher die $k$-n�chsten-Nachbarn von $\mathbf{x}_i$ enth�lt hat somit ein festes Volumen. 
	Wir skalieren $g$ mit der inversen Distanz zum $k$-ten Nachbarn, dann gilt f�r das Volumen von $B$, 
	$V(B(\mathbf{x}_i)) = \frac{\pi^{\frac{n}{2}}}{\Gamma(\frac{n}{2}+1)}$, somit ist auch die zweite Bedingung aus Lemma \ref{lem:geod} 
	f�r unsere Daten $X$ erf�llt. 

	\begin{rem}
		Dabei ist der \textit{$j$-te Nachbar von $\mathbf{x}_i$ bzgl. $d$} gegeben durch $\mathbf{x}_{i_j}$, so dass 
		$d(\mathbf{x}_i, \mathbf{x}_{i_1}) \leq \dots \leq d(\mathbf{x}_i,\mathbf{x}_{i_j}) \leq \dots \leq d(\mathbf{x}_i,\mathbf{x}_{i_N})$. 
		Die $k$-n�chsten-Nachbarn eines Punktes sind somit die $1\text{-},\dotsc,k$-ten-Nachbarn. 
	\end{rem}

	Wir k�nnen nun f�r jedes $\mathbf{x}_i$ einen metrischen Raum $X_i$ definieren, so dass die Distanz zu den $k$-n�chsten-Nachbarn die 
	Geod�te auf der riemannschen Mannigfaltigkeit ist. 
	Sei $d$ die zu unseren Daten geh�rende Metrik. Dann definieren wir f�r $\mathbf{x}_i \in X$ den metrischen Raum $(X, \tilde{d}_i)$ mit 

	\begin{align}
		\tilde{d}_i(x,y) \coloneqq \frac{d(x,y)}{k_{x_i}},
		\label{eq:tilde}
	\end{align}
	
	dabei bezeichnet $k_{x_i}$ den $k$-ten Nachbarn von $\mathbf{x}_i$. Diese Definition der $d_i$ ist f�r den Kontext nicht sinnvoll, 
	da f�r $\mathbf{x}_i$ mit $h$-ten Nachbarn $\mathbf{x}_h$ und $j$-ten Nachbarn $\mathbf{x}_j$, mit $h,j \leq k$, nach Lemma \ref{lem:geod} 
	$\tilde{d}_i$ nur f�r die Paare $(\mathbf{x}_i, \mathbf{x}_h), (\mathbf{x}_i, \mathbf{x}_j)$ die Geod�te angibt. Wir setzen, 

	\begin{align}
		\bar{d}_i(x,y) \coloneqq \begin{cases}
									\tilde{d}_i(x,y), &\quad \text{falls } x = \mathbf{x}_i \vee y = \mathbf{x}_i, \\
									\infty, 		  &\quad \text{sonst.}
					  			 \end{cases}
		\label{eq:bar}
	\end{align}

	Somit sind die $\bar{d}_i$ erweiterte Metriken. 

	Eine bekannte Problematik, wenn man hochdimensionale Daten betrachtet ist der \textit{Fluch der Dimensionen}. Dieses Ph�nomen 
	beschreibt die Effekte der Volumenvergr��erung in hochdimensionalen R�umen. 
	Um zwei Auswirkungen auf paarweise Distanzen zu beschreiben, betrachten wir die paarweisen Distanzen randomisierter gleichverteilter 
	Punkte in $n$-dimensionalen euklidischen R�umen, siehe Abbildung \ref{fig:Curse}. Die liefert uns (1) mit zunehmender Gr��e der Dimension 
	erh�hen sich die paarweisen Distanzen, (2) dass die paarweisen Distanzen sind ungef�hr normalverteilt, wobei die Varianz der Normalverteilung 
	f�r h�here Dimensionen abnimmt. Dadurch sind die Distanzen eines Punktes zu seinen ersten, zweiten, \dots, k-ten Nachbarn im hochdimensionalen 
	Raum ann�hernd gleich. F�r eine genauere Analyse der Auswirkungen hochdimensionaler R�ume auf die n�chsten Nachbarn siehe \cite{Beyer}. 

	\begin{figure}
		%\centering
		\includegraphics[width=400px, height=178px]{Figures/pairwise_dist}
		%\decoRule
		\caption[Durchsch. Distanz]{Paarweise Distanzen von $N=500$ zuf�llig gleichverteilten Punkten im $R^D$.}
		\label{fig:Curse}
	\end{figure}

	Unter anderem kann man dem \textit{Fluch entfliehen}, in dem die Distanzen mit der Distanz zum ersten Nachbarn normalisiert werden. %TODO: f�ge Grafik ein 
	Dies wenden wir auf unsere erweiterten Metriken $\bar{d}_i$ an und erhalten erweiterte Pseudometriken, 
	
	\begin{align}
		d_i(\mathbf{x}_i,\mathbf{x}_j) \coloneqq \max(0, \bar{d}_i(\mathbf{x}_i,\mathbf{x}_j) - \bar{d}_i(\mathbf{x}_i,\mathbf{x}_{i_1})).
		\label{eq:di}
	\end{align}

	\begin{rem}
		Wir nehmen an, dass unsere Daten $X$ keine Duplikate enthalten. Diese Annahme ist gerechtfertigt, da wir prim�r Aussagen �ber die 
		Beziehung zwischen den Datenpunkten treffen m�chten. Der erste Nachbar ist also ein \textit{echter Nachbar}, mit 
		$\bar{d}_i(\mathbf{x}_i,\mathbf{x}_{i_1}) > 0$. 
	\end{rem}

	\begin{rem}
		F�r den Fall, das die Metrik der zugrundeliegenden Mannigfaltigkeit $d_\M$ bekannt ist, setzen wir in Gleichung (\ref{eq:tilde}) 
		$\tilde{d}_i \coloneqq d_\M$ und wenden die Modifikationen aus Gleichungen (\ref{eq:bar}) und (\ref{eq:di}) an um $d_i$ zu erhalten. 
	\end{rem}

	Die erweiterten Pseudometriken $d_i$ liefern uns lokal die Geod�te, welche hilfreich ist die zugrundeliegende Mannigfaltigkeit zu beschreiben. 
	Allerdings sind die Metriken nicht zwingend miteinander kompatibel. Eine L�sung f�r die Inkompatibilit�t der Metriken werden wir im folgenden Abschnitt beschreiben.

%----------------------------------------------------------------------------------------

\section{Topologische Repr�sentation} \label{sec:repr}
	In Satz \ref{thm:adj} haben wir gesehen, dass es eine Adjunktion zwischen topologischen R�umen und simplizialen Mengen gibt. 
	Wir k�nnten die in Gleichung (\ref{eq:di}) definierten Metriken als topologische R�ume mit $\{(X,\tau_i)\}_{1 \leq i \leq N}$ und 
	der von $d_i$ induzierten Topologie $\tau_i$ auffassen, diese mittels singul�re Mengen Funktors in simpliziale Mengen �berf�hren 
	und die Mengen Vereinigen. Durch diese Konstruktion w�rden uns wichtige Informationen verloren gehen. Um dies zu vermeiden, 
	werden wir eine Adjunktion zwischen der Kategorie der erweiterten pseudo-metrischen R�ume $\EPMet$ und der Kategorie der 
	unscharfen simplizialen Mengen $\sFuzz$ konstruieren.

	\begin{rem}
		Da wir nur endliche Datens�tze betrachten, werden wir uns auf die \textit{Unterkategorien} der endlichen erweiterten pseudo-metrischen R�ume 
		$\fEPMet$ und endlichen unscharfen simpliziale Mengen $\fsFuzz$ beschr�nken. Eine Unterkategorie besteht aus Teilmengen der Objekte und Morphismen 
		der zugeh�rigen Kategorie. 
	\end{rem}

	\begin{defn}
		Der Funktor $\fReal : \fsFuzz \to \fEPMet$ ist gegeben durch

		\begin{align}
			\fReal(X) \coloneqq \varinjlim \fReal(\Delta^n_{<a}),
		\end{align}

		wobei,

		\begin{align}
			 \quad \fReal(\Delta^n_{<a}) &\coloneqq (\{x_1,\dotsc, x_n\}, d_a), \\
			 \quad d_a(x_i,x_j) &\coloneqq \begin{cases}
			 								   0  			&\text{, falls } i = j \\
			 								   -\log(a)		&\text{, sonst.}
			 							  \end{cases}
		\end{align}

		Die Wirkung des Funktors $\fReal$ auf einem Morphismus $\Delta^n_{<a} \to \Delta^m_{<b}$, mit $a \leq b$ und $\sigma: \Delta^n \to \Delta^m$, 
		ist gegeben durch $(\{x_1,\dotsc,x_n\},d_a) \mapsto (\{x_{\sigma(1)},\dotsc,x_{\sigma(n)}\},d_b)$.
	\end{defn}

	% \begin{rem}
	% 	Dabei betrachten wir �hnlich zu Satz \ref{thm:adj} den Limes �ber \dots %TODO
	% \end{rem}

	\begin{rem}
		Der Funktor ist wohldefiniert, da aus $a \leq b, d_a \geq d_b$ folgt. Somit ist die Wirkung von $\fReal$ auf einem Morphismus von $\fsFuzz$ 
		nichtexpansiv und somit ein Morphismus von $\fEPMet$.
	\end{rem}

	\begin{thm}
		Die Funktoren $\fReal$ und $\fSing : \fsFuzz \to \fEPMet$, wobei f�r $Y \in \fEPMet$ gilt,

		\begin{align}
			\fSing(Y) : ([n], [0,a)) \to \hom_\fEPMet (\fReal(\Delta^n_{<a}), Y),
		\end{align}

		sind zueinander adjungiert.
	\end{thm}

	% TODO: Intuition

	\begin{rem}
		Ein Beweis findet sich in \cite{UMAP}. Die wesentliche Idee ist dabei, dass Funktoren welche Limiten erhalten einen rechts adjungierten 
		Funktor besitzen, nach Konstruktion erh�lt $\fReal$ Limiten. Zus�tzlich wird f�r den Beweis das Yoneda Lemma und Gleichung (\ref{eq:colimit}) 
		verwendet. 
	\end{rem}

	Die konstruierte Adjunktion erm�glicht es uns nun die erweiterten pseudo-metrischen R�ume $\{(X,d_i)\}_{1 \leq i \leq N}$, 
	mit $d_i$ aus Gleichung (\ref{eq:di}), mittels des $\fSing$ Funktors als unscharfe simpliziale Mengen darzustellen. 
	Diese verkn�pfen wir mittels t-Conorm und erhalten die \textit{unscharfe topologische Repr�sentation} des Datensatzes $X$

	\begin{align}
		\tConorm \fSing((X, d_i)).
		\label{eq:toprep}
	\end{align}

	%-----------------------------------

	\subsection*{Intuition der Repr�sentation}

	Der folgende Absatz soll die Konstruktion der unscharfen topologischen Repr�sentation intuitiv erkl�ren. Dabei werden wir an einigen Stellen 
	mathematische Strenge gegen eine illustrative Herangehensweise eintauschen.	

	Lokale Eigenschaften der Daten werden dadurch erhalten, dass wir mit Lemma \ref{lem:geod} die Geod�ten auf der $X$ zugrundeliegenden 
	Mannigfaltigkeit bestimmt haben. F�r die erweiterten pseudo-metrischen R�ume, wird dann eine geeignete unscharfe simpliziale Menge konstruiert. 
	In Abschnitt \ref{seq:sSet} haben wir argumentiert, dass es intuitiv oft gen�gt anstelle einer simplizialen Menge einen Simplizialkomplex zu betrachten. 
	Deshalb k�nnen wir uns den Funktor $\fSing$ als \textit{Abbildung} des erweiterten pseudo-metrischen Raumes $X_i$ auf einen unscharfen 
	Simplizialkomplex $\mathcal{K}_i$ vorstellen, wobei jeder Simplex einen Zugeh�rigkeitsgrad hat. 

	Betrachten wir nun das $1$-Skelett von $\mathcal{K}_i$, so sind die $0$-Simplizes die $\mathbf{x}_j \in X$ mit Zugeh�rigkeitsgrad $1$. 
	Die $1$-Simplizes beschreiben Abst�nde zwischen den $\mathbf{x}_j$, wobei der Zugeh�rigkeitsgrad eines $1$-Simplizes aus $\mathcal{K}_i$, mit Facetten 
	$\mathbf{x}_j, \mathbf{x}_l$, gerade $\exp(-d_i(\mathbf{x}_j, \mathbf{x}_l))$ entspricht. Der Zugeh�rigkeitsgrad des ersten Nachbarn von $\mathbf{x}_i$ 
	in $\mathcal{K}_i$ ist also stets $1$ und nimmt f�r weiter entfernte Nachbarn exponentiell ab. Die Repr�sentation erh�lt also metrische Informationen 
	der $\mathbf{x}_i$, bevorzugt dabei allerdings stark die lokalen Abst�nde. Dies spiegelt die Aussage des Lemma \ref{lem:geod} wieder, dass wir nur lokal 
	zu $\mathbf{x}_i$ die Geod�te bestimmen k�nnen. $2$-Simplizes w�rden Eigenschaften �ber die Fl�che zwischen drei Punkten erhalten, $3$-Simplizes 
	�ber das Volumen welches von 4 Punkten eingeschlossen ist. Simplizes h�herer Ordnung die entsprechenden Analogon f�r h�her dimensionale R�ume. 

	Um die lokal unterschiedlichen Zugeh�rigkeitsgrade der $1$-Simplizes % TODO:

	Die Vereinigung der unscharfen simplizialen Mengen in Gleichung (\ref{eq:toprep}) l�sst sich wie folgt f�r den stark vereinfachten Fall des $1$-Skelett 
	der Simplizialkomplexe $\mathcal{K}_i$ beschreiben. Die $0$-Simplizes bleiben unver�ndert. 

	Wir werden sp�ter, in Kapitel \ref{Implementierung}, auf diese Interpretation der $1$-Simplizes zur�ckkommen. 
	Doch zuerst m�chten wir noch beschreiben, wie die unscharfe topologische Repr�sentation f�r das UMAP Verfahren genutzt wird um 
	eine niedrigdimensionale Darstellung der Daten zu finden.

%----------------------------------------------------------------------------------------

\section{Einbettung} \label{sec:einb}
	In diesem Abschnitt soll eine 

	Um die niedrigdimensionale Repr�sentation der hochdimensionalen anzupassen wird f�r das UMAP Verfahren die 
	Kreuzentropie zwischen zwei unscharfen (simplizialen) Mengen vorgeschlagen. 
	
	\begin{equation}
		C((A,\mu), (A, \nu)) \coloneqq \sum_{a \in A} \left(\mu(a)\log\left(\frac{\mu(a)}{\nu(a)}\right) +  \left(1-\mu(a)\right)\log\left(\frac{1-\mu(a)}{1-\nu(a)}\right)  \right)
		\label{eq:crossentropy}
	\end{equation}
	
	In Abschnitt \ref{seq:numerische_formulierung} werden wir die Kreuzentropie genauer betrachten.

%---------------------------------------------------------------------------------------- 
%!TEX root = ../main.tex

\chapter{Implementierung} 

\label{Implementierung}

% Docs for algorithms: 
% http://mirror.physik-pool.tu-berlin.de/pub/CTAN/macros/latex/contrib/algorithmicx/algorithmicx.pdf

%----------------------------------------------------------------------------------------

%\newcommand{\cech}{\u{C}ech}  % Symbol for Cech complex

%----------------------------------------------------------------------------------------
%----------------------------------------------------------------------------------------

% TODO: Beschreibung der L�cke in diesem Kapitel oder in UMAP
% \section{Einleitung}
In diesem Kapitel m�chten wir die Implementierung des UMAP Verfahren beschreiben. 
Die vollst�ndige Berechnung aller Simplizes hat eine exponnentielle Laufzeit, da hierf�r 
alle Teilmengen unseres $N$-elementigen Datensatzes betrachtet werden m�ssten. 
In der Implementierung von Leland McInnes et. al. \cite{cpu}  werden hingegen nur alle 
zwei-elementigen Teilmengen betrachtet. 
Wie wir in Kapitel \ref{Experimente} sehen werden, liefert uns diese Approximation des 
\cech-Komplexes sehr gute visuelle Ergebbnisse. 
Zuerst werden wir den Algorithmus mit seinen Subroutinen in Pseudo-Code angeben. Danach 
werden wir Ans�tze nennen um die L�cke zwischen Theorie und Praxis zu schlie�en. 
Zus�tzlich werden wir die rechenintensiven Schritte des Verfahrens betrachten und eine 
effizientere Implementierung auf der GPU betrachten.

%----------------------------------------------------------------------------------------

% N O T I Z E N
% Cite Numba 24

% T E X T B A U S T E I N E
% 

%----------------------------------------------------------------------------------------
%----------------------------------------------------------------------------------------

\section{Pseudo-Code}

Das berechnen des 
\begin{algorithm}
\caption{UMAP algorithm}
\label{umap}
\begin{algorithmic}
\Function{UMAP}{$X, N, D, d, min\_dist, n\_epochs$}  %description
	\For{$x \in X$}
		\State $knn(x) \gets k\text-NearestNeighbour(x)$
		\State $graph(x) \gets \bigcup_{y \in knn(x)} (\{x, y\}, exp(-d_{x,y})) \ \cup \ \bigcup_{y \notin knn(x)} (\{x, y\}, 0)$ % \Comment{union weights by using prob. t-norm ($w_{x,y} + w_{y,x} - w_{x,y} * w_{y,x}$)}
   \EndFor
   \State $A \gets \text{weighted adjacency matrix}(\bigcup_{x \in X} graph(x))$
   \State $D \gets \text{degree matrix for the graph } A$
   \State $L \gets D^{1/2} (D-A) D^{1/2}$ \Comment{Symmetric normalized Laplacian}
   \State $evec \gets \text{sorted Eigenvectors of } L$
   \State $Y \gets evec[1,...,d\text{+}1]$
   \State $Y \gets OptimizeEmbedding(Y, min\_dist, n\_epochs)$
   \State \textbf{return} $Y$
\EndFunction
\end{algorithmic}
\end{algorithm}

% Plot wie die stetige Approximierung f�r verschiedene Parameter a und b aussieht.

% Einer der beiden rechenintensivsten Subroutinen im UMAP Algorithmus ist die n�chste Nachbar suche.
% in \cite{Tang} (k-NN is bottleneck) werden 3 verschiedene Methoden verglichen. Sehr effizient ist auch die 
% Facebook FAISS Methode.



%!TEX root = ../main.tex

\chapter{Experimente} 

\label{Experimente}

%----------------------------------------------------------------------------------------

% A U F B A U
	% Welche Verfahren
	% Welche Daten
	% Welche Ergebnisse unter welchen Parametern
	% Welche Laufzeit
	% Welche Schlussfolgerungen

	% N O T I Z E N
	% Cartoonset, MNIST, (WIND)
	% PCA, UMAP, t-SNE, LargeVis

	% Test transform function

	% Google News f�r n_samples Dimension

	% Outlier, add completly different images 

	% Erw�hne analytische Bewertungen der Einbettungen
	% trustworthiness-continuity (Kaski et al., 2003)
	% mean (smoothed) precision-recall (Venna et al., 2010)
	% nearest-neighbor accuracy (Van Der Maaten et al., 2009)

	% Stabilit�t der Ergebnisse

	% Komplett neue Gesichter

	% Cluster on FMNIST, Scores on MNIST (Costfunction and SVD)

	% Auswirkung von sampling auf kleinen Datens�tzen

	% T E X T B A U S T E I N E
	% 'Empirisches testen der Hypothesen/ Annahmen'
	% 'UMAP auf realen/echten Daten'
	% In \cite{UMAP} wurde bereits die Stabilit�t des Verfahrens getestet.
	% Insbesondere verweisen wir auf Abbildung 3,4 in Abschnitt 5.
	% Die Stabilit�t des UMAP Verfahrens ist 

	% Mehr Datenpunkte, dann h�here perplexit�t f�r gleichen Ausgang, �hnlich zu n_neighbor
	% tSNE vergr��ert Regionen mit h�herer Dichte

	% Bei cartoonpca784 bleibt 99,842% der Varianz erhalten.
	% Bei cartoonpca10k weniger als 10^-5 * 2 verlust der Varianz 

	% Um eine gute Interpretation der Daten zu geben ist es wichtig ein sehr gutes Verst�ndnis 
	% der Datens�tze zu besitzen, deshalb haben wir uns auf leicht interpretierbare Daten beschr�nkt

	% Verweise auf Daten und erw�hne deutlich feinere Struktur von UMAP
	% https://www.biorxiv.org/content/biorxiv/early/2018/04/10/298430.full.pdf

	% Keine Wordembeddings, da die Vorvererbeitung sehr wichtig ist.

%----------------------------------------------------------------------------------------
%----------------------------------------------------------------------------------------

% \section{Einleitung}
	Nach ausf�hrlicher Darstellung der Theorie des UMAP Verfahrens, m�chten wir nun UMAP auf 
	drei Datens�tzen mit alternativen Verfahren empirisch testen.
	Wir werden eine m�glichst vollst�ndige Darstellung der Ergebnisse in dieser Arbeit pr�sentieren. 
	Allerdings ist es zu empfehlen die Ergebnisse in einem interaktiven Jupyter notebook zu betrachten. 
	Dieses befindet sich auf der beigelegten CD oder auf GitHub
	\footnote{\url{https://github.com/reinerschristopher/bachelorthesis}}. %!!! TODO: korrigiere link
 	
 	% TODO: schreibe Aufbau

%----------------------------------------------------------------------------------------

\section{Alternative Verfahren} \label{sec:verfahren}
	Wir haben uns dazu entschieden UMAP mit t-SNE, TriMap und PCA zu vergleichen. Die ersten beiden 
	Verfahren sollen kurz eingef�hrt werden, wobei wir insbesondere die wichtigen Hyperparameter der 
	Verfahren angeben. Eine Kenntnis des PCA Verfahrens setzen wir voraus. 
	An dieser Stelle ist zu bemerken, dass in den letzten Jahren zahlreiche neue DR Verfahren entwickelt wurden. 
	Aufgrund dessen k�nnen wir in dieser Arbeit nur eine kleine Teilmenge der DR Verfahren betrachten. 
	Ein sehr ausf�hrlicher Vergleich findet sich in \cite{Maaten2008}. %, dieser ber�cksichtigt allerdings keine neueren Verfahren, da er 2008 publiziert wurde. 


	Da die Implementierungen des Laplacian Eigenmaps Verfahrens und Isomap Verfahrens schlecht f�r gro�e 
	Datens�tze skalieren, haben wir uns bewusst dazu entschieden, diese nicht mit in die Analyse aufzunehmen. 

	%-----------------------------------

	\subsection{t-SNE}
	% TODO: wichtige Terme einf�gen?
	Das t-SNE Verfahren \cite{tSNE} ist zur Zeit eines der bekanntesten und meistgenutzten 
	nicht-linearen Dimensionsreduktionsverfahren. Dabei wird t-SNE fast ausschlie�lich 
	zur Visualisierung genutzt, da die Laufzeit f�r h�here Einbettungsdimensionen schlecht ist. 
	% TODO: Wie wir sp�ter sehen werden. (oder unten?)

	t-SNE konstruiert zuerst eine Wahrscheinlichkeitsverteilung $P$ auf Paaren $(i, j)$ der hochdimensionalen 
	Datenpunkten. Diese ist so gew�hlt, dass Paare �hnlicher Objekte eine h�here Wahrscheinlichkeit 
	zugeordnet bekommen, wohingegen sehr unterschiedliche Datenpunkte eine Wahrscheinlichkeit nahe $0$ haben. 
	Die �hnlichkeit der Punkte wird dabei meist mittels der euklidischen Distanz gemessen, 
	kann aber �hnlich wie im UMAP Algorithmus durch andere Metriken ersetzt werden. Um $P$ zu konstruieren 
	wird eine Gau�verteilung genutzt, wobei die Varianz abh�ngig vom \code{perplexity} Parameter ist. 
	Die so erhaltenen Wahrscheinlichkeiten $p_{i | j}$ sind im Allgemeinen nicht symmetrisch. 
	Die Symmetrie wird durch mitteln der Daten erhalten. 

	�hnlich wird eine Wahrscheinlichkeitsverteilung $Q$ im niedrigdimensionalen Raum mithilfe der 
	studentschen t-Verteilung konstruiert. Urspr�nglich wurde $Q$ ebenfalls durch eine Gausverteilung 
	konstruiert, das so erhaltene Verfahren (SNE \cite{tSNEvar1}) ist allerdings aufgrund einer 
	schwierig zu optimierenden Zielfunktion und dem \enquote{crowding problem} wenig praktikabel.

	Um die d-dimensionale Repr�sentation der Daten zu optimieren wir die Kullback-Leiber Divergenz 
	von zwischen $P$ und $Q$ bez�glich der $y_i$ minimiert. % TODO: werden y_i in Einleitung erw�hnt?

	Seit der Ver�ffentlichung des Verfahrens wurden zahlreiche Verbesserungen, 
	insbesondere f�r die Laufzeit, vorgeschlagen \cite{tSNEmod1,tSNEmod3}. 
	Dabei ist besonders Barnes-Hut-SNE \cite{tSNEmod2} zu erw�hnen, allerdings sollte hier beachtet werden, 
	dass aufgrund der Konstruktion einer speziellen Datenstruktur die Laufzeit f�r $d>3$ sehr schlecht ist.

	Die von t-SNE produzierte Repr�sentation der Daten ist vom \code{perplexity} Parameter abh�ngig. 
	Dabei kann man festhalten, je gr��er die \code{perplexity} ist, desto gr��er ist die Varianz  % TODO: Verweise auf die Gleichungen oben 
	des Gau�verteilung. Somit werden f�r gro�e \code{perplexity} Werte globalere Strukturen erfasst, 
	da der Gau�kern sehr breit ist. Wenn der \code{perplexity} Parameter in der Gr��enordnung der 
	Anzahl an Datenpunkten $N$ liegt gleicht t-SNE dem MDS Verfahren. % TODO: Referenz? 

	Der zweite wichtige Hyperparameter, welchen wir beschreiben m�chten ist die \code{exaggeration}.
	meistens wird hier zwischen \code{early-} und \code{late-exaggeration} unterschieden. 
	Im wesentlichen verbessert der Parameter die Optimierung des Gradienten und sorgt daf�r, 
	dass Punkte desselben Clusters m�glichst schnell in der niedrigdimensionalen Repr�sentation 
	gruppiert werden \cite{tSNEcluster}. Der \code{late-exaggeration} wie in \cite{tSNEmod3} 
	beschrieben kontrahiert gefundene Cluster, so lassen sich in einer 2- oder 3-dimensionalen 
	Darstellung leichter Cluster bestimmen - entweder visuell oder mittels Clustering-Verfahren. 

	Wir werden reale Datens�tze analysieren und das Verhalten der Hyperparameter beschreiben, um 
	ein zus�tzliches Verst�ndnis f�r die von t-SNE genutzten Hyperparameter zu bekommen 
	empfiehlt sich \cite{tSNEparam}, dort werden interaktiv auf k�nstlich erzeugten Datens�tzen 
	die Auswirkung der Parameter gezeigt.

	F�r unsere Experimente haben wir die scikit Implementierung des t-SNE Verfahren genutzt \cite{scikit-learn}.
	Zus�tzlich haben wir die openTSNE \cite{opentSNE} Implementierung genutzt. Diese beschleunigt 
	die Laufzeit des t-SNE Algorithmus durch eine zus�tzliche Fouriertransformation \cite{tSNEmod3}.
	Die openTSNE Implementierung besitzt im Vergleich zur scikit Implementierung die M�glichkeit 
	den \code{late-exaggeration} Parameter zu spezifizieren. % TODO: sp�ter sehen wir die 

	Um einen Laufzeitvergleich mit der GPU Implementierung des UMAP Verfahrens zu erm�glichen, 
	werden wir eine GPU Implementierung von t-SNE betrachten \cite{tSNEGPUcode}. 
	Eine Beschreibung findet sich in \cite{tSNEGPUpaper}.

	%-----------------------------------

	\subsection{TriMap}
	Das TriMap Verfahren \cite{TriMap} soll eine globalere Repr�sentation der Daten finden als 
	beispielsweise t-SNE, da nicht nur paarweise die �hnlichkeit zweier Objekte $i, j$ betrachtet wird, 
	sondern stets Triple $i, j, k$. Die so erhaltene globale Struktur der Daten soll die 
	Cluster-Abst�nde der Daten repr�sentieren. Die von uns gew�hlte Implementierung des Verfahrens findet 
	sich in \cite{trimapcode}.

	% TODO: TriMap besser beschreiben. 

	Wir haben diesem Algorithmus gew�hlt, da die Tripletts �hnlichkeiten mit 2-Simplizes des UMAP Verfahren haben.
	Die Tripletts sind vergleichbar mit den 2-Simplizes des UMAP Verfahrens. 
	Insbesondere k�nnen die Ans�tze eine lineare Teilmenge ($O(N)$) an Tripletts zu finden 
	weitere Entwicklungen des UMAP Verfahren motivieren.

%----------------------------------------------------------------------------------------

% TODO: !!! Komplett �berarbeiten
\section{Bewertung der Ergebnisse} \label{sec:bwertung}
	Um die $d$-dimensionalen Repr�sentationen der Daten zu bewerten gibt es verschiedene Ans�tze. 
	Diese sollen nun vorgestellt und verglichen werden. % !!TODO: entweder scores erg�nzen oder text �ndern

	Ein DR-Verfahren, welches die statistischen Eigenschaften der zugrundeliegenden Daten erfasst, 
	sollte invariant bez�glich Rauschen der Daten sein. Hingegen sollte eine schlechte Einbettung 
	wenig stabil sein, wenn zus�tzliches Rauschen in den Daten auftritt. % TODO: Score angeben

	In \cite{Harmeling} werden zwei Methoden zur Auswahl eines geeigneten DR-Verfahrens verglichen. 

	Um die lokale Qualit�t der Algorithmen zu analysieren haben wir uns die \textit{Cluster} in der 
	$2$-dimensionalen Repr�sentation angeschaut, wobei wir als Cluster eine Teilmenge der Daten bezeichnen, 
	welche deutlich von den anderen Datenpunkten getrennt ist. % TODO: 'Stetigkeit' innerhalb der Cluster?
	Insbesondere bei der Analyse des Cartoon Set \ref{sec:cartoonset} konnten wir gut lokale Strukturen 
	erkennen, da jeder Datenpunkt mehrere Eigenschaften gegeben hat -- im Vergleich zum MNIST und FMNIST 
	Datensatz, wo uns nur ein \textit{Label} pro Datenpunkt gegeben ist. 

	Die globale Struktur der Repr�sentation qualitativ zu bewerten ist subjektiv. Dabei ist insbesondere 
	die Frage -- \enquote{Wie \textit{stark} unterscheiden sich die Cluster?} -- zu beantworten. 

	F�r die qualitative Analyse wird die F�higkeit des Gehirns genutzt Strukturen zu erkennen. 
	Nachteile der qualitativen Analyse sind, (1) die Subjektivit�t und somit Abh�ngigkeit vom Betrachter, 
	(2) dass sie nur im $d$-dimensionalen $(d \leq 3)$ m�glich ist. 
	% und (3) die Problematik, dass es (noch) keine mathematische Formulierung der Bewertung gibt.

	In den folgenden Experimenten haben wir uns auf eine qualitative Analyse der Daten beschr�nkt.
	Einerseits ist dies unserer Erfahrung nach (und Eintr�gen in Internetforen, \ldots) die 
	meistgenutzte Art die Repr�sentation in der Praxis zu bewerten. Zus�tzlich bietet die Wahl der 
	Datens�tze eine gute Gelegenheit die Repr�sentationen visuell zu bewerten. 

%----------------------------------------------------------------------------------------

\section{Cartoon Set} \label{sec:cartoonset}
	In diesem Abschnitt werden wir den \textit{Cartoon Set} Datensatz analysieren \cite{cartoon}. 
	Dabei werden wir:
	\begin{itemize}
		\item sehen, dass UMAP eine vergleichbare Laufzeit mit der von FIT-SNE hat
		\item das Verhalten der niedrigdimensionalen Darstellung unter verschiedenen 
			  Hyperparametern betrachten
		\item eine exemplarische Beschreibung der Hyperparameter geben
		\item UMAP mit anderen Dimensionsreduktionsverfahren vergleichen
		\item sehen, dass UMAP zugrundeliegende Mannigfaltigkeiten erkennt und darstellt
	\end{itemize}

	%-----------------------------------

	\subsection*{Beschreibung des Datensatzes}
	Der Cartoon Datensatz enth�lt \num{100000} unterschiedliche Bilder von gezeichneten Gesichtern 
	(siehe Abbildung \ref{fig:Cartoon_Sample}). 

	\begin{figure}
		%\centering
		\includegraphics[width=400px, height=73px]{Figures/Cartoon_Sample}
		%\decoRule
		\caption[Ausschnitt des Cartoon Set]{Sechs zuf�llig gew�hlte Gesichter des Cartoon Set.}
		\label{fig:Cartoon_Sample}
	\end{figure}

	Die Bilder wurden aus 16 Labels zusammengesetzt (u.a. Gesichtsform, Gesichtsfarbe, Frisur, Haarfarbe), 
	dabei variiert die Anzahl der M�glichkeiten pro Label zwischen zwei (Augenlid, Wimpern,\dots) und 
	111 (Anzahl m�gliche Frisuren). Die Farben der Komponenten wurden aus einem diskreten RGB Raum gew�hlt. 
	Insgesamt ergibt sich eine m�gliche Anzahl von $10^{13}$ Gesichtern. F�r die Analyse haben wir verschiedene 
	Eigenschaften zusammengefasst um einen besseren �berblick zu haben. 
	Beispielsweise haben wir die 111 Frisuren, nach qualitativer Analyse, zu 19 Frisurformen zusammengefasst. 

	Die urspr�ngliche Gr��e eines Bildes betrug $500 \times 500$ Pixel mit vier Farbkan�len 
	(CYMK-Darstellung der Farben). Aufgrund des gro�en Randes haben wir uns dazu entschieden die Gr��e der 
	Bilder auf $300 \times 300$ ohne nennenswerten Informationsverlust zu verringern. Somit betr�gt die Dimension 
	des Cartoon Set $D = \num{360000}$ und die Anzahl an Beispielen $N$ variiert zwischen $\num{10000}$ und $\num{100000}$.

	Wir haben uns f�r diesen Datensatz entschieden um UMAP auf Daten mit einer komplexeren Struktur zu testen 
	als dies in \cite{UMAP} gemacht wird. Dabei ist auch zu beachten, dass aufgrund der 16 Labels aus 
	welchen die Gesichter bestehen, die Qualit�t einer Einbettung schwieriger zu beurteilen ist als beispielsweise 
	im MNIST Datensatz (siehe Abschnitt \ref{sec:MNIST}). Wir haben die Bewertung der Einbettung unter der Annahme 
	gemacht, dass \textit{�hnliche} Gesichter \textit{�hnliche} Hautfarben, Frisuren, Haarfarben, Brillen und B�rte 
	besitzen. Diese f�nf Eigenschaften m�chten wir besonders hervorheben, da sie die dominantesten Merkmale des Gesichts 
	beschreiben. Somit werden wir eine Einbettung des Cartoon Set als \textit{gut} bewerten, wenn sie zwischen 
	diesen f�nf Merkmalen unterscheidet.

	% Histogram �ber Verteilung der Frisuren

	% unterschiedliche n_components (range(15))

	% gute ergebnisse wegen aaprox kNN und lokaler Zusammenhang

	% Erw�hne hohe Umbegungsdimension

	%-----------------------------------

	\subsection*{Qualitative Analyse der Ergebnisse}
		% Globale Struktur sehr gut erhalten und lokale auch, wenig abh�ngig von Parametern


	\begin{figure} % PCA als Startwert, global sind hair, face_color und hair_color, nicht faicial_hair und glasses
		%\centering
		\includegraphics[width=400px, height=200px]{Figures/pca_cartoon}
		%\decoRule
		\caption[PCA auf Cartoon Set]{PCA auf dem Cartoon Set, mit $N=\num{50000}, D=\num{360000}$. 
									Dabei sind die Punkte bez�glich der folgenden Attribute gekennzeichnet (v.l.n.r): Frisurtyp, Gesichtsfarbe, Haarfarbe.}
		\label{fig:pca_cartoon}
	\end{figure}

	\begin{figure} % Globale struktur der Punkte in den Verfahren. cmap=nhair
		%\centering
		\includegraphics[width=400px, height=131px]{Figures/cartoon_cluster}
		%\decoRule
		\caption[Cluster des Cartoon Set]{(v.l.n.r) UMAP, t-SNE, TriMap auf dem Cartoon Set, mit $N=\num{10000}, D=\num{360000}$. 
										Dabei entsprechen die Punkte des blauen Gebiets dem oberen Drittel der PCA Einbettung, 
										das gelbe Gebiet dem unteren Drittel der PCA Einbettung und die Punkte au�erhalb eines Gebietes dem mittleren Teil der PCA Einbettung}
		\label{fig:cartoon_cluster}
	\end{figure}

	\begin{figure} % Lokale struktur der Punkte in den Verfahren. cmap=nhair_color and cmap=nface_color. �berg�nge in TriMap flie�ender, tSNE weniger geh�uft/dicht als umap
		%\centering
		\includegraphics[width=400px, height=272px]{Figures/cartoon_local}
		%\decoRule
		\caption[Cluster des Cartoon Set]{(v.l.n.r) UMAP, t-SNE, TriMap auf dem Cartoon Set, mit $N=\num{10000}, D=\num{360000}$. 
										blablabla}
		\label{fig:cartoon_local}
	\end{figure}

	\begin{figure} % Lokale struktur der Punkte in den Verfahren, cluster sehr �hnlich. cmap=nhair_color
		%\centering
		\includegraphics[width=400px, height=134px]{Figures/cartoon_local_local}
		%\decoRule
		\caption[Cluster des Cartoon Set]{(v.l.n.r) UMAP, t-SNE, TriMap auf dem Cartoon Set, mit $N=\num{10000}, D=\num{360000}$. 
										blablabla}
		\label{fig:cartoon_local_local}
	\end{figure}

	\begin{figure}
		%\centering
		\includegraphics[width=400px, height=278px]{Figures/umap_10k_vs_50k}
		%\decoRule
		\caption[UMAP auf 50k und 10k Daten]{TODO: Beschreibung des Bildes}
		\label{fig:umap_10k_vs_50k}
	\end{figure}



%----------------------------------------------------------------------------------------

\section{MNIST} \label{sec:MNIST}
	% Lokal gut Einbettung vergleichbar mit anderen Verfahren
	% variiere set_opt_mix
	% Beispielsweise sollten die Cluster welche die Ziffern 1 und 7 darstellen n�her zueinander liegen als die Cluster der Ziffern 1 und 6. 
%----------------------------------------------------------------------------------------

\section*{FMNIST} \label{sec:FMNIST}
	% Flie�ender �bergang zwischen Clustern, "Fehler" sind begr�ndet

%----------------------------------------------------------------------------------------

\section{Laufzeitanalyse} \label{sec:runtime}
	Die praktischen Tests der Verfahren wurden auf Rechnern mit einer Linux-Architektur ausgef�hrt. 
	Die CPU Tests haben wir auf Intel Xeon 6136 CPUs mit 48 Kernen und 384 GB RAM ausgef�hrt. 
	F�r die Verfahren welche mittels Berechnungen auf einer Graphikkarte verbessert wurden, haben 
	wir Intel Xeon Gold 6136 CPUs mit 188 GB RAM und Nvidia V100 GPUs genutzt. 
	Insgesamt haben wir �ber 100 Experimente gemacht um genauere Aussagen �ber die Laufzeit der Verfahren 
	zu treffen und diese in Abh�ngigkeit der wichtigen Parameter zu setzen.

%----------------------------------------------------------------------------------------

\section{Zusammenfassung der Ergebnisse}
	Die Ergebnisse haben gezeigt, dass alle drei verwendeten DR Verfahren lokale Strukturen der 
	Bilddatens�tze erkannt haben. 

%----------------------------------------------------------------------------------------
%!TEX root = ../main.tex

\chapter{Ausblick} 

\label{Ausblick}

%----------------------------------------------------------------------------------------

% Anwendung auf andere Datens�tze Beispielsweise Kundendatenbank, als Vorverarbeitung, 
% Zeitreihenanalyse

% H�here Simplizes betrachten (siehe Methoden von TriMap)

% Verfahren zum finden der zugrundeliegenden Dimension, kann variieren (=> variierender 
% near_neighbor Parameter?)

% UMAP zur Gesichtserkennung (siehe Eigenfaces)

%----------------------------------------------------------------------------------------
%----------------------------------------------------------------------------------------

%\begin{equation}
\begin{align*}
	&\qquad&
	\sum_{i=1}^n \exp\left(\frac{-(\text{knn-dists}_i - \rho)}{\sigma}\right) &= \log_2(n)	\\
	\iff&& \sum_{i=1}^n (\exp(-(\text{knn-dists}_i - \rho)) - \exp(\sigma)) &= \log_2(n) \\
	\iff&& \left(\sum_{i=1}^n \exp(-(\text{knn-dists}_i - \rho))\right) - n\exp(\sigma) &= \log_2(n) \\
	\iff&& \sum_{i=1}^n \exp(-(\text{knn-dists}_i - \rho)) &= \log_2(n) + n\exp(\sigma) \\
	\iff&& \log_e \left(\frac{\sum_{i=1}^n \exp(-(\text{knn-dists}_i - \rho)) - \log_2(n)}{n}\right) &= \sigma \\
\end{align*}
%\end{equation}
%!TEX root = ../main.tex

\chapter{Zusammenfassung} 

\label{Zusammenfassung}

%----------------------------------------------------------------------------------------

% Problem mit rauschenden Daten: nutze random projections (~siehe Tang)

% Anwendung wo Mannigfaltigkeitshypothese nicht wahr ist siehe Oudot s.68
% Bildausschnitte, Sensordaten,...

%----------------------------------------------------------------------------------------
%---------------------------------------------------------------------------------------- 

%----------------------------------------------------------------------------------------
%	THESIS CONTENT - APPENDICES
%----------------------------------------------------------------------------------------

\appendix % Cue to tell LaTeX that the following "chapters" are Appendices

% Include the appendices of the thesis as separate files from the Appendices folder
% Uncomment the lines as you write the Appendices

%%!TEX root = ../main.tex

\chapter{Appendix A}

\label{AppendixA}

%----------------------------------------------------------------------------------------

% TODO: evtl Cech und VR in Grundlagen
\section{Homologie}
	% Hom�omorphismus, Homologie (Gruppen), homologie Equivalenz,
	Eine bekannte Aussage, welche der Topologie entstammt, ist, dass eine Kaffeetasse das Gleiche \textit{topologische Objekt} 
	wie ein Donut ist. Dies ist dadurch begr�ndet, dass man eine Kaffeetasse durch strecken und stauchen 
	(ohne rei�en) in einen Donut transformieren kann. Diese \textit{Gleichheit} l�sst sich wie folgt mathematisch darstellen:

	\begin{defn}[Stetige Abbildung]
		Sei ...
	\end{defn}

	\begin{defn}[Hom�omorphismus]
		Sei ...
	\end{defn}

	Nun werden wir den Begriff der Homologie einf�hren. Daf�r werden wir den Begriff der simplizialen Komplexe 
	ben�tigen um zuerst die simpliziale Homologie einzuf�hren.

	\begin{defn}[Simplizialer Komplex]
		Sei ...
	\end{defn}

	\begin{defn}[Abstrakter simplizialer Komplex]
		Sei ...
	\end{defn}

	\begin{ex}
		\cech-Komplex
	\end{ex}

	\begin{ex}
		VR-Komplex
	\end{ex}

	\begin{defn}[Homologie]
		Sei ...
	\end{defn}

	\begin{defn}[Homologiegruppe]
		Sei ...
	\end{defn}

	\begin{ex}
		Die Homologiegruppen eines Balls, Donuts, Graphen, ...
	\end{ex}

	\begin{defn}[Filtration]
		Sei $T \subseteq \R$, eine \textit{Filtration} $\mathcal{X}$ �ber $T$ ist eine Familie von 
		topologischen R�umen $\{X_i\}_{i \in T}$, so dass $X_i \subseteq X_j$ 
		f�r $i \leq j \in T$.  % (Oudot s.29)
	\end{defn}

	\begin{rem}
		Eine Filtration $\mathcal{X}$ wird meist nicht als Familie verschiedener topologischer 
		R�ume $X_i$ angesehen, sondern als ein einziger Raum, welcher sich im Laufe der Zeit 
		\enquote{transformiert}. Siehe dazu \cite{Oudot}.  % (Oudot s.30)
		Dies stimmt mit unserem Bild �berein, den $\epsilon$-Parameter eines \cech-Komplexes %TODO: Satz l�schen und motivation cech komplex nennen
		immer gr��er werden zu lassen. Wir werden \textit{persistente Homologie} nutzen um 
		die homologischen Eigenschaften zu beschreiben, welche f�r alle $i \in T$ erhalten 
		(bzw. persistent) bleiben.
	\end{rem}

	% TODO: Oudot Kap. 4,5 f�r Anwendungen von persistenter Homologie

	% TODO: Bemerkung, dass \mathcal{X} eine Repr�sentation eines Poset ist (siehe Oudot s.29)

	\begin{ex}
		% TODO: warum?
		Insbesondere ist somit der \cech-Komplex eine Filtration, da ...
	\end{ex}

	\begin{ex}
		% TODO: warum?
		Eine weitere Filtration ist der Vietoris-Rips-Komplex
	\end{ex}

	\begin{rem} % TODO: Bem in Def umwandeln
		Der VR-Komplex ist ein Beispiel eines Clique-Komplexes und wird als solcher 
		eindeutig durch sein 1-Skelet identifiziert. Sei $V$ ein VR-Komplex und $V^1$ das 
		zugeh�rige 1-Skelet. Dann erh�lt man $V$ aus $V^1$, indem man alle Cliquen im 
		$V^1$ zugrundeliegenden Graphen bestimmt. Eine k-Clique ist ein vollst�ndiger Subgraph 
		mit k Knoten. 
	\end{rem}

	Es gibt weitere Formen von Komplexen, die bekanntesten sind die CW-Komplexe und Zellkomplexe. 
	Ziel ist es stets mit Hilfe einfacher Konstrukte die Topologie eines Raums zu beschreiben.

	Die vom \cech-Komplex und vom Vietoris-Rips-Komplex beschriebene Homologie kann sich unterscheiden (siehe Abbildung \ref{fig:Cech_VR}).

	\begin{figure}
		%\centering
		\includegraphics[width=400px, height=274px]{Figures/Cech_VR_complex}  % width=1\textwidth
		%\decoRule
		\captionsetup{justification=RaggedRight}  % centerfirst
		\caption[Versch. Komplexe]{Eine Punktwolke [oben links] kann zu einem \cech-Komplex [unten links] 
								   oder einem VR-Komplex [unten rechts] basierend auf dem Parameter $\epsilon$ 
								   konstruiert werden. Die Homotopietypen der $\epsilon/2$ �berdeckung vom 
								   \cech-Komplex ($S^1 \vee S^1 \vee S^1$), w�hrend das VR-Komplex die Homotopietypen 
								   ($S^1 \vee S^2$) besitzt. (\textit{Quelle:} \cite{Barcodes})}

		\label{fig:Cech_VR}
	\end{figure}

	Wir ben�tigen nun noch den Begriff einer \textit{guten �berdeckung} um dann eine geeignete Aussage �ber einen \cech-Komplex einer Menge zu machen. 

	\begin{defn}[�berdeckung]
		Eine �berdeckung eines topologischen Raumes ist \dots
		Eine gute �berdeckung ist \dots
	\end{defn}

	\begin{NT}  % Nerv Theorem
		% Ein top. Raum ist homotopie equivalent zu einer endlichen guten �berdeckung. 
		Sei ...
		\label{thm:NT}
	\end{NT}
	
	Das Nerv Theorem \ref{thm:NT} liefert uns eine Aussage dar�ber, dass ein topologischer Raum $X$ zu einer endlichen guten 
	�berdeckung von $X$ homotopie�quivalent ist.
	Wie wir sehen ist die Konstruktion stark abh�ngig vom $\epsilon$-Parameter. Um dem 
	entgegenzuwirken nutzt man die Idee der \textit{persistenten Homologie}. Dabei 
	wird betrachtet wie sich die Homologie eines Raumes f�r eine monoton wachsende Folge 
	$(\epsilon_i)_{i \in \N}$ verh�lt.

	F�r den von uns betrachteten Fall, die Struktur einer Riemannschen Mannigfaltigkeit $(\M, g)$ zu beschreiben, werden wir 
	Annehmen, dass die Daten $\mathbf{x}_i \in X$ gleichverteilt bez�glich der Riemannmetrik $g$ sind, bzw. $g$ so w�hlen, das 
	die Annahme erf�llt ist. Somit erhalten wir eine gute �berdeckung ohne eine Abh�ngigkeit von $\epsilon$ zu haben. 

%%!TEX root = ../main.tex

\chapter{Appendix B} \label{AppendixB}


%----------------------------------------------------------------------------------------


   % Um eine Unterscheidung zwischen der theoretischen Sichtweise auf das UMAP Verfahren, welche alle $k$-Simplizes 
   % $(1 \leq k \leq N)$ ber�cksichtigt, und der praktischen Implementierung zu verdeutlichen, werden wir die 
   % verwendete Notation anpassen. 

   % Die unsicheren Mengen $(A, \mu), (A, \nu)$ aus Gleichung (\ref{eq:crossentropy}) lassen sich als gewichtete Graphen, 
   % in Form einer Adjazenzmatrix darstellen. % TODO: F�r den Fall, dass nur 1-Simplizes betrachtet werden
   % Das 1-Skelett der hochdimensionalen Repr�sentation notieren wir als Adjazenzmatrix $V$ mit $\vij = \mu(a)$, f�r 
   % ein 1-Simplex $a$, mit Facetten $i$ und $j$. 
   % % TODO: Wahl von \vij

   % Die Wahl des Zugeh�rigkeitsgrades f�r 1-Simplizes der niedrigdimensionalen Repr�sentation der Daten ist wie folgt gegeben:

   % \begin{equation}
   %    \wij = (1+a(\norm{\mathbf{y}_i - \mathbf{y}_j}^2_2)^b)^{-1}, \label{eq:wij} % TODO: wie sind a, b gew�hlt?
   % \end{equation}

   % % TODO: f�r a=b=1 tSNE, b=1 LargeVis

   % wobei $\mathbf{y}_i, 1 \leq i \leq N$ die Vektoren der Einbettung sind. Die Wahl der Werte $a,b$ werden wir in 
   % Abschnitt \ref{seq:hyper} beschreiben.


   % Somit erhalten wir folgende Kreuzentropie zwischen den Graphen der hoch- und niedrigdimensionalen 
   % Darstellung der Daten:

   % \begin{alignat}{2}
   %    C(V, W) &= \sumtup{i}{j} \vij \log\left(\frac{\vij}{\wij}\right) + (1-\vij) \log\left(\frac{1-\vij}{1-\wij}\right) \\
   %            &= \begin{aligned}[t]
   %                   &  \sumtup{i}{j} (\vij \log(\vij) + (1-\vij) \log(1-\vij)) \\
   %                   &- \sumtup{i}{j} (\vij \log(\wij)) - \sumtup{i}{j} ((1-\vij) \log(1-\wij))
   %                \end{aligned} \\
   %            &=  C_v - \sumtup{i}{j} (\vij \log(\wij) + (1-\vij) \log(1-\wij)) \label{eq:loss}
   % \end{alignat}

   % Der Term $C_v$ bleibt w�hrend der Minimierung der Funktion bez�glich der $\mathbf{y}_i$ konstant.
   
   % Aufgrund der Wahl einer differenzierbaren Funktion f�r den Zugeh�rigkeitsgrad der 1-Simplizes, 
   % bzw. der Gewichte des niedrigdimensionalen Graphen, l�sst sich nun der Gradient der Zielfunktion (\ref{eq:loss}) herleiten.

   % Mittels der Kettenregel ergibt sich mit $d_{ij} \coloneqq \norm{\mathbf{y}_i - \mathbf{y}_j}_2$:

   % \begin{align}
   % \quad
   %    \deriv{C}{\mathbf{y}_i} &= \sumtup{k}{l} \deriv{C}{w_{kl}} \sumtup{m}{n} \deriv{w_{kl}}{d^2_{mn}} \sumtup{p}{q} \deriv{d^2_{mn}}{d_{pq}} \deriv{d_{pq}}{\mathbf{y}_i} \\
   %                   &= \sumtup{k}{l} \deriv{C}{w_{kl}} \sumtup{m}{n} \deriv{w_{kl}}{d^2_{mn}} \deriv{d^2_{mn}}{d_{mn}} \deriv{d_{mn}}{\mathbf{y}_i} \\
   %                   &= 2 \sumtup{k}{l} \deriv{C}{w_{kl}} \sumtups{m} \deriv{w_{kl}}{d^2_{mi}} \deriv{d^2_{mi}}{d_{mi}} \deriv{d_{mi}}{\mathbf{y}_i} \\
   %                   &= 2 \sumtups{m} \left( \sumtup{k}{l} \deriv{C}{w_{kl}} \deriv{w_{kl}}{d^2_{mi}} \right) \deriv{d^2_{mi}}{d_{mi}} \deriv{d_{mi}}{\mathbf{y}_i}  \label{eq:deriv-2}\\
   %                   &= 4 \sumtups{m} \left( \sumtup{k}{l} \deriv{C}{w_{kl}} \deriv{w_{kl}}{d^2_{mi}} \right) (\mathbf{y}_i - \mathbf{y}_m) \label{eq:deriv-1}
   % \end{align}

   % Bei der Umformung von (\ref{eq:deriv-2}) nach (\ref{eq:deriv-1}) haben wir verwendet, dass $d_{mi}$ die euklidische Norm ist. 
   % F�r Einbettungen in andere metrische R�ume m�sste der Gradient an dieser Stelle entsprechend angepasst werden. Da wir 
   % das UMAP Verfahren im wesentlichen zur Visualisierung im $\R^2$ benutzen ist die Wahl der euklidischen Norm gerechtfertigt.
   % Die �brigen Umformungen ergeben sich aus umordnen und wegfallen der Terme. 

   % (\ref{eq:deriv-1}) l�sst sich weiter umformen, dazu nutzen wir:

   % \begin{align}
   %    \deriv{C}{\wij} &= - \frac{\vij}{\wij} + \frac{1-\vij}{1-\wij} = \frac{\wij - \vij}{\wij (1 - \wij)} \\
   %    \deriv{\wij}{d^2_{ij}} &= -bad_{ij}^{2(b-1)} \wij^2 = - \frac{b}{d_{ij}^2} \wij (1 - \wij)
   % \end{align}

   % Der UMAP Gradient ist also gegeben durch:
   % % TODO: Daniel: Fehlt da nicht noch eine innere Summe aus 4.9?
   % \begin{equation}
   %    \deriv{C}{\mathbf{y}_i} = 4 \sumtups{j} (\wij - \vij) \frac{b}{d_{ij}^2} (\mathbf{y}_i - \mathbf{y}_j)
   %    \label{eq:umapgrad}
   % \end{equation}
%\include{Appendices/AppendixC}

%----------------------------------------------------------------------------------------
%	BIBLIOGRAPHY
%----------------------------------------------------------------------------------------

\printbibliography[heading=bibintoc]

%----------------------------------------------------------------------------------------

%----------------------------------------------------------------------------------------
%	HELP
%----------------------------------------------------------------------------------------
%
% Bibliography not showing? Run pdflatexmk
%


\end{document}  
