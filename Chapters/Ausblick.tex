%!TEX root = ../main.tex

\chapter{Ausblick} 

\label{Ausblick}

%----------------------------------------------------------------------------------------

% Anwendung auf andere Datensätze Beispielsweise Kundendatenbank, als Vorverarbeitung, 
% Zeitreihenanalyse

% Höhere Simplizes betrachten (siehe Methoden von TriMap)

% Verfahren zum finden der zugrundeliegenden Dimension, kann variieren (=> variierender 
% near_neighbor Parameter?)

% UMAP zur Gesichtserkennung (siehe Eigenfaces)

% “Product quantization for nearest neighbor search”, Jegou & al., PAMI 2011.
% Zur Darstellung hochdimensionaler Vektoren
% This can be seen as a lossy compression technique for high-dimensional vectors, 
% that allows relatively accurate reconstructions and distance computations in the compressed domain.

% Betrachten einer Umkehrabbildung

%----------------------------------------------------------------------------------------
%----------------------------------------------------------------------------------------

%\begin{equation}
\begin{align*}
	&\qquad&
	\sum_{i=1}^n \exp\left(\frac{-(\text{knn-dists}_i - \rho)}{\sigma}\right) &= \log_2(n)	\\
	\iff&& \sum_{i=1}^n (\exp(-(\text{knn-dists}_i - \rho)) - \exp(\sigma)) &= \log_2(n) \\
	\iff&& \left(\sum_{i=1}^n \exp(-(\text{knn-dists}_i - \rho))\right) - n\exp(\sigma) &= \log_2(n) \\
	\iff&& \sum_{i=1}^n \exp(-(\text{knn-dists}_i - \rho)) &= \log_2(n) + n\exp(\sigma) \\
	\iff&& \log_e \left(\frac{\sum_{i=1}^n \exp(-(\text{knn-dists}_i - \rho)) - \log_2(n)}{n}\right) &= \sigma \\
\end{align*}
%\end{equation}