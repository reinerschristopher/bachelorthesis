%!TEX root = ../main.tex

\chapter{Zusammenfassung und Ausblick} \label{Zusammenfassung}

%----------------------------------------------------------------------------------------
%----------------------------------------------------------------------------------------

\section{Zusammenfassung} \label{sec:sum}

	In dieser Arbeit haben wir eine kompakte Einf�hrung in die Datenanalyse gegeben und 
	dabei betont, dass Daten, insbesondere Daten hoher Dimensionen, eine immer wichtigere 
	Rolle spielen, da sie in gro�en Mengen gesammelt werden k�nnen. 
	Wir haben eine Klasse an Verfahren vorgestellt, welche zu gegebenen hochdimensionalen 
	Daten eine Einbettung in einem niedrigdimensionalen Raum finden. 
	
	Um das UMAP Verfahren vorzustellen, haben wir Aussagen aus unterschiedlichen 
	mathematischen Teilgebieten kennengelernt. Dabei haben wir einerseits einen 
	Fokus auf die mathematische Korrektheit gelegt, andererseits mittels Erkl�rungen 
	und Beispielen die Anschauung hinter den eingef�hrten Konzepten erl�utert. 

	Die Theorie der unscharfen topologischen Repr�sentation wurde eingef�hrt und 
	mit Erkl�rungen erg�nzt. Es wurde eine Beschreibung der Implementierung 
	gegeben und diese anschlie�end genutzt, um Datens�tze zu analysieren. Zudem haben 
	wir die Problemstellung betrachtet, ein Qualit�tsma� f�r Einbettungen zu finden,
	und einen neuen Datensatz ausf�hrlich analysiert. 

	Dabei hoffen wir, dass wir den Leser f�r das UMAP Verfahren und die 
	(topologische) Datenanalyse begeistern und ihn zu weiteren 
	�berlegungen motivieren konnten. 

% Problem mit rauschenden Daten: nutze random projections (~siehe Tang)

% Anwendung wo Mannigfaltigkeitshypothese nicht wahr ist siehe Oudot s.68
% Bildausschnitte, Sensordaten,...

% Keine genaue Interpretation über Dimensionen

% Lokale Eigenschaften sind wichtiger als globale

% Metric Learning 54, 7

% Durch lokale Eigenschaft sind die resultierenden Cluster ähnlich groß, 
% unabhängig von der Ausgangsgröße

% Was sind die stärken was sind die schwächen?

% Aufgrund der schnellen Implementierung auf dem GPU ist eine Interaction mit dem Nutzer 
% möglich \cite{Verleysen} sieht dies als wichtigen Punkt für DR Algorithmen
% (siehe https://perso.uclouvain.be/michel.verleysen/papers/iconip13mv.pdf)
% gemeinsam mit online kNN
% (siehe: https://arxiv.org/pdf/1804.03032.pdf)

% Froh die einen Einblick in die TDA bekommen zu haben auch wenn nur sehr klein.

%----------------------------------------------------------------------------------------

\section{Ausblick} \label{sec:ausblick}
	In den vergangenen Monaten haben wir uns intensiv mit dem UMAP Verfahren 
	auseinandergesetzt. Einige weiterf�hrende �berlegungen sollen nun vorgestellt werden.  

	Zun�chst m�chten wir eine Problematik aufgreifen, welche bei der 
	Entwicklung des UMAP Verfahren auftritt. In Kapitel \ref{UMAP} wurde die Theorie 
	des Verfahrens entwickelt. Die dort verwendeten Aussagen, um die Theorie des UMAP 
	Verfahrens mathematisch zu begr�nden, bauen auf unsicheren simplizialen Mengen auf. 
	Die aktuellen praktischen Implementierungen greifen dabei nicht auf diese Mengen 
	zur�ck, sondern auf eine (starke) Vereinfachung. 
	Es sollte genauer untersucht werden, wie \textit{stark} diese Vereinfachung in der 
	Praxis ist. In dieser Arbeit haben wir begonnen, Ans�tze daf�r zu 
	entwickeln, aber die daf�r notwendigen Kenntnisse der Topologie h�tten den 
	Rahmen dieser Arbeit gesprengt. F�r unsere Ans�tze haben 
	wir \cech-Komplexe und VR-Komplexe betrachtet. Diese werden in der topologischen 
	Datenanalyse an verschiedenen Stellen verwendet. Kurz gesagt, kann man sich unter 
	dem VR-Komplex den konstruierten Graphen und unter einem \cech-Komplex die unscharfe 
	topologische Repr�sentation $\Xrep$ vorstellen, dies ist nicht ganz richtig, da 
	beide Konstrukte nur Simplizialkomplexe sind und nicht die allgemeinere Form der 
	simplizialen Mengen darstellen. 
	Diese Herangehensweise w�rde die Theorie der Praxis angleichen. 
	Umgekehrt k�nnte man auch die praktische Implementierung auf $n$-Simplizes erweitern. 
	Ein interessanter Ansatz ist dabei die Wahl der Tripel in \cite{TriMap}. 

	Eine weiteres Problem ist die quantitative Bewertung von Einbettungen. 
	Wie wir gesehen haben stellt diese eine schwierige Aufgabe dar. 

	Im Rahmen dieser Arbeit haben wir uns auf Bilddaten beschr�nkt. Wir m�chten den 
	Leser dazu motivieren, das UMAP Verfahren auf anderen Datens�tzen zu analysieren. 

	Wir haben uns nicht mit der Frage nach der intrinsischen Dimension besch�ftigt, doch 
	diese Frage ist keineswegs trivial. Insbesondere spielt die Frage eine Rolle, 
	wenn UMAP zur Vorverarbeitung von Daten genutzt wird. Dabei k�nnen wir uns vorstellen, 
	dass sich die Theorie der simplizialen Mengen daf�r als n�tzlich erweisen kann, da 
	eine simpliziale Menge im Wesentlichen aus unterschiedlich-dimensionalen Simplizes besteht.  
	Somit k�nnten verschiedene Bereiche der Daten eine unterschiedliche Dimension annehmen. 

	Wir m�chten die Arbeit mit folgendem Zitat abschlie�en, 

	\vspace*{0.01\textheight}
	\flushright \noindent \textit{\enquote{The goal is to turn data into information, and information into insight.}} \bigbreak
	\hfill Carly Fiorina

%----------------------------------------------------------------------------------------
%----------------------------------------------------------------------------------------