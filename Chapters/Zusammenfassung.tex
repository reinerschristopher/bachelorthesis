%!TEX root = ../main.tex

\chapter{Zusammenfassung und Ausblick} 

\label{Zusammenfassung}

%----------------------------------------------------------------------------------------

% Problem mit rauschenden Daten: nutze random projections (~siehe Tang)

% Anwendung wo Mannigfaltigkeitshypothese nicht wahr ist siehe Oudot s.68
% Bildausschnitte, Sensordaten,...

% Keine genaue Interpretation über Dimensionen

% Lokale Eigenschaften sind wichtiger als globale

% Metric Learning 54, 7

% Durch lokale Eigenschaft sind die resultierenden Cluster ähnlich groß, 
% unabhängig von der Ausgangsgröße

% Was sind die stärken was sind die schwächen?

% Aufgrund der schnellen Implementierung auf dem GPU ist eine Interaction mit dem Nutzer 
% möglich \cite{Verleysen} sieht dies als wichtigen Punkt für DR Algorithmen
% (siehe https://perso.uclouvain.be/michel.verleysen/papers/iconip13mv.pdf)
% gemeinsam mit online kNN
% (siehe: https://arxiv.org/pdf/1804.03032.pdf)

%----------------------------------------------------------------------------------------

% Anwendung auf andere Datensätze Beispielsweise Kundendatenbank, als Vorverarbeitung, 
% Zeitreihenanalyse

% Höhere Simplizes betrachten (siehe Methoden von TriMap)

% Verfahren zum finden der zugrundeliegenden Dimension, kann variieren (=> variierender 
% near_neighbor Parameter?)

% UMAP zur Gesichtserkennung (siehe Eigenfaces)

% “Product quantization for nearest neighbor search”, Jegou & al., PAMI 2011.
% Zur Darstellung hochdimensionaler Vektoren
% This can be seen as a lossy compression technique for high-dimensional vectors, 
% that allows relatively accurate reconstructions and distance computations in the compressed domain.

% Betrachten einer Umkehrabbildung

% Erwähne smallvis (und 'schnellere' uwot version) für SGD ohne sub sampleing
% Da wir uns in den Experiementen auf die DR mittels schneller Implementierungen für 
% große Datensätze fokussiert haben, sind diese beiden Implementierungen nicht erwähnt..

% TODO: Beschreibung der Lücke in diesem Kapitel oder in UMAP

% Wie kann man Daten besser vorverarbeiten um die Ergebnisse noch zu verbessern?

%----------------------------------------------------------------------------------------
%----------------------------------------------------------------------------------------