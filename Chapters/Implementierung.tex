%!TEX root = ../main.tex

\chapter{Implementierung} 

\label{Implementierung}

% Docs for algorithms: 
% http://mirror.physik-pool.tu-berlin.de/pub/CTAN/macros/latex/contrib/algorithmicx/algorithmicx.pdf

%----------------------------------------------------------------------------------------

%\newcommand{\cech}{\u{C}ech}  % Symbol for Cech complex

%----------------------------------------------------------------------------------------
%----------------------------------------------------------------------------------------

% TODO: Beschreibung der L�cke in diesem Kapitel oder in UMAP
% \section{Einleitung}
In diesem Kapitel m�chten wir die Implementierung des UMAP Verfahren beschreiben. 
Die vollst�ndige Berechnung aller Simplizes hat eine exponentielle Laufzeit, da hierf�r 
alle Teilmengen unseres $N$-elementigen Datensatzes betrachtet werden m�ssten. 
In der Implementierung von Leland McInnes et. al. \cite{cpu}  werden hingegen nur alle 
zweielementigen Teilmengen betrachtet. 
Wie wir in Kapitel \ref{Experimente} sehen werden, liefert uns diese Approximation des 
\cech-Komplexes sehr gute visuelle Ergebnisse.  % Cech Komplex?
Zuerst werden wir den Algorithmus mit seinen Subroutinen in Pseudo-Code angeben. Danach 
werden wir Ans�tze nennen um die L�cke zwischen Theorie und Praxis zu schlie�en. 
Zus�tzlich werden wir die rechenintensiven Schritte des Verfahrens betrachten und eine 
effizientere Implementierung auf der GPU betrachten.

%----------------------------------------------------------------------------------------

% N O T I Z E N
% Cite Numba 24

% T E X T B A U S T E I N E
% 

%----------------------------------------------------------------------------------------
%----------------------------------------------------------------------------------------

\section{Pseudo-Code}

Das berechnen des 
\begin{algorithm}
\caption{UMAP Algorithmus}
\label{umap}
\begin{algorithmic}
\Function{UMAP}{$X, N, D, d, min\_dist, n\_epochs$}  %description
	\For{$x \in X$}
		\State $knn(x) \gets k\text-NearestNeighbour(x)$
		\State $graph(x) \gets \bigcup_{y \in knn(x)} (\{x, y\}, exp(-d_{x,y})) \ \cup \ \bigcup_{y \notin knn(x)} (\{x, y\}, 0)$ % \Comment{union weights by using prob. t-norm ($w_{x,y} + w_{y,x} - w_{x,y} * w_{y,x}$)}
   \EndFor
   \State $A \gets \text{weighted adjacency matrix}(\bigcup_{x \in X} graph(x))$
   \State $D \gets \text{degree matrix for the graph } A$
   \State $L \gets D^{1/2} (D-A) D^{1/2}$ \Comment{Symmetric normalized Laplacian}
   \State $evec \gets \text{sorted Eigenvectors of } L$
   \State $Y \gets evec[1,...,d\text{+}1]$
   \State $Y \gets OptimizeEmbedding(Y, min\_dist, n\_epochs)$
   \State \textbf{return} $Y$
\EndFunction
\end{algorithmic}
\end{algorithm}

% Plot wie die stetige Approximierung f�r verschiedene Parameter a und b aussieht.


%----------------------------------------------------------------------------------------

\section{Profiling}
   In Kapitel \ref{Experimente} werden wir genauer auf die tats�chliche Laufzeit des UMAP Algorithmus eingehen. 
   An dieser Stelle m�chten wir die rechenintensiven Subroutinen des Verfahrens ausmachen. Daf�r haben wir den 
   Python cProfiler verwendet, dieser misst die Laufzeit der aufgerufenen Funktionen. 
   Um f�r verschiedene Umgebungsdimensionen vergleichbare Ergebnisse zu erhalten, haben wir $N = \num{10000}$ 
   Datenpunkte in $10$ unterschiedlichen $D = [100, 500, 1000, 5000, 10000, 50000]$-dimensionalen Gau�-verteilten 
   Datenwolken gew�hlt. F�r den Funktionsaufruf haben wir die Standardparameter verwendet, insbesondere haben wir 
   die Daten in einen zweidimensionalen Raum eingebettet.
   Dabei ist uns aufgefallen, dass besonders der k-n�chste-Nachbarn-Algorithmus und die Optimierung mittels 
   stochastischem Gradienten Verfahren einen gro�en Teil der Laufzeit des Verfahrens beanspruchen. In Tabelle 
   \ref{table:profiling} sind die Ergebnisse der Profilierung zusammengefasst. 

   \begin{table}
   \begin{tabular}{l|ll}
   D     & Laufzeit NN-Descent & Laufzeit der Optimierung \\ \hline
   100   & 9\%                 & 75,3\%                   \\
   500   & 12\%                & 73,8\%                   \\
   1000  & 14\%                & 72,9\%                   \\
   5000  & 30,4\%              & 58\%                     \\
   10000 & 44\%                & 45,1\%                   \\
   50000 & 78,8\%              & 14,8\%                  
   \end{tabular}
   \caption{$D$ beschreibt die Gr��e der Umgebungsdimension. Abh�ngig von D haben wir die Laufzeit des UMAP Verfahrens 
            profiliert. Die zweite und dritte Spalte beziehen sich auf die relativen Laufzeiten des kNN Verfahrens und 
            der Optimierung der Einbettung.}
   \label{table:profiling}
   \end{table}

   Insbesondere scheint der \textit{NN-Descent} Algorithmus \cite{k-NNG} die Laufzeit des UMAP Verfahren f�r 
   hochdimensionale Daten stark zu beeinflussen. 

%----------------------------------------------------------------------------------------

\section{N�chste-Nachbarn-Klassifikation}
% Einer der beiden rechenintensivsten Subroutinen im UMAP Algorithmus ist die n�chste Nachbar suche.
% in \cite{Tang} (k-NN is bottleneck) werden 3 verschiedene Methoden verglichen. Sehr effizient ist auch die 
% Facebook FAISS Methode.
   Zum effizienten finden der 1-Simplizes der topologischen Repr�sentation unserer Daten, ben�tigen wir einen 
   k-n�chste-Nachbarn-Algorithmus (kurz: \textit{kNN-Algorithmus}). 
   %TODO: Beschreibung was ein kNN-Alg macht

   Das Ergebnis eines kNN-Algorithmus wird meist in einem ungerichteten Graph -- dem kNN-Graph -- dargestellt, 
   wobei die Knoten den Datenpunkten entsprechen und die Kanten den Nachbarschaftsbeziehungen, 
   somit besitzt jeder Knoten Grad k.

   Bei einer naiven Implementierung betr�gt die Laufzeit $\mathcal{O}(N^2D)$ (wobei N die Anzahl der Datenpunkte  % TODO: warum N^2D?
   und D die Dimension der Datenpunkte ist). Mit einer effizienten Implementierung ist in der Praxis eine 
   ann�hernd in N lineare Laufzeit m�glich. Die Herangehensweisen lassen sich nach \cite{Tang} in drei Kategorien 
   einteilen. (1) Baum basierte Verfahren auf Partitionen des Raumes, (2) Hashfunktionen auf lokalen Teilgebieten des Raumes 
   (3) Nachbarschafts-Erkundungen. 

   Wir m�chten nun drei Verfahren vorstellen. 
   \subsection*{NN-Descent}
      Der NN-Descent Algorithmus \cite{k-NNG} beruht auf dem Prinzip der Nachbarschafts-Erkundungen. Dabei wird ein 
      initialer kNN-Graph iterativ verbessert, unter der Annahme, dass die Nachbarschaftsbeziehung 
      transitiv ist, f�r zwei vorhandene Nachbarschaftspaare $(x, y), (y,z)$ also mit hoher Wahrscheinlichkeit auch 
      ein Nachbarschaftspaar $(x,z)$ im kNN-Graph existiert. 


      Ein Vorteil des NN-Descent Verfahren ist, dass kein globaler Index der verwaltet werden muss. Somit ist eine 
      Anwendung auf gro�en Datens�tzen m�glich welche nicht komplett in den Arbeitsspeicher (RAM) des verwendeten Rechners 
      geladen werden k�nnen. 







