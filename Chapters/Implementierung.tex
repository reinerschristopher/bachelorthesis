%!TEX root = ../main.tex

\chapter{Implementierung} 

\label{Implementierung}

% Docs for algorithms: 
% http://mirror.physik-pool.tu-berlin.de/pub/CTAN/macros/latex/contrib/algorithmicx/algorithmicx.pdf

%----------------------------------------------------------------------------------------

%\newcommand{\cech}{\u{C}ech}  % Symbol for Cech complex

%----------------------------------------------------------------------------------------
%----------------------------------------------------------------------------------------

% TODO: Beschreibung der L�cke in diesem Kapitel oder in UMAP
% \section{Einleitung}
In diesem Kapitel m�chten wir die Implementierung des UMAP Verfahren beschreiben. 
Die vollst�ndige Berechnung aller Simplizes hat eine exponnentielle Laufzeit, da hierf�r 
alle Teilmengen unseres $N$-elementigen Datensatzes betrachtet werden m�ssten. 
In der Implementierung von Leland McInnes et. al. \cite{cpu}  werden hingegen nur alle 
zwei-elementigen Teilmengen betrachtet. 
Wie wir in Kapitel \ref{Experimente} sehen werden, liefert uns diese Approximation des 
\cech-Komplexes sehr gute visuelle Ergebbnisse. 
Zuerst werden wir den Algorithmus mit seinen Subroutinen in Pseudo-Code angeben. Danach 
werden wir Ans�tze nennen um die L�cke zwischen Theorie und Praxis zu schlie�en. 
Zus�tzlich werden wir die rechenintensiven Schritte des Verfahrens betrachten und eine 
effizientere Implementierung auf der GPU betrachten.

%----------------------------------------------------------------------------------------

% N O T I Z E N
% Cite Numba 24

% T E X T B A U S T E I N E
% 

%----------------------------------------------------------------------------------------
%----------------------------------------------------------------------------------------

\section{Pseudo-Code}

Das berechnen des 
\begin{algorithm}
\caption{UMAP algorithm}
\label{umap}
\begin{algorithmic}
\Function{UMAP}{$X, N, D, d, min\_dist, n\_epochs$}  %description
	\For{$x \in X$}
		\State $knn(x) \gets k\text-NearestNeighbour(x)$
		\State $graph(x) \gets \bigcup_{y \in knn(x)} (\{x, y\}, exp(-d_{x,y})) \ \cup \ \bigcup_{y \notin knn(x)} (\{x, y\}, 0)$ % \Comment{union weights by using prob. t-norm ($w_{x,y} + w_{y,x} - w_{x,y} * w_{y,x}$)}
   \EndFor
   \State $A \gets \text{weighted adjacency matrix}(\bigcup_{x \in X} graph(x))$
   \State $D \gets \text{degree matrix for the graph } A$
   \State $L \gets D^{1/2} (D-A) D^{1/2}$ \Comment{Symmetric normalized Laplacian}
   \State $evec \gets \text{sorted Eigenvectors of } L$
   \State $Y \gets evec[1,...,d\text{+}1]$
   \State $Y \gets OptimizeEmbedding(Y, min\_dist, n\_epochs)$
   \State \textbf{return} $Y$
\EndFunction
\end{algorithmic}
\end{algorithm}

% Plot wie die stetige Approximierung f�r verschiedene Parameter a und b aussieht.

% Einer der beiden rechenintensivsten Subroutinen im UMAP Algorithmus ist die n�chste Nachbar suche.
% in \cite{Tang} (k-NN is bottleneck) werden 3 verschiedene Methoden verglichen. Sehr effizient ist auch die 
% Facebook FAISS Methode.


