%!TEX root = ../main.tex

\chapter{Appendix B} \label{AppendixB}


%----------------------------------------------------------------------------------------


   % Um eine Unterscheidung zwischen der theoretischen Sichtweise auf das UMAP Verfahren, welche alle $k$-Simplizes 
   % $(1 \leq k \leq N)$ ber�cksichtigt, und der praktischen Implementierung zu verdeutlichen, werden wir die 
   % verwendete Notation anpassen. 

   % Die unsicheren Mengen $(A, \mu), (A, \nu)$ aus Gleichung (\ref{eq:crossentropy}) lassen sich als gewichtete Graphen, 
   % in Form einer Adjazenzmatrix darstellen. % TODO: F�r den Fall, dass nur 1-Simplizes betrachtet werden
   % Das 1-Skelett der hochdimensionalen Repr�sentation notieren wir als Adjazenzmatrix $V$ mit $\vij = \mu(a)$, f�r 
   % ein 1-Simplex $a$, mit Facetten $i$ und $j$. 
   % % TODO: Wahl von \vij

   % Die Wahl des Zugeh�rigkeitsgrades f�r 1-Simplizes der niedrigdimensionalen Repr�sentation der Daten ist wie folgt gegeben:

   % \begin{equation}
   %    \wij = (1+a(\norm{\mathbf{y}_i - \mathbf{y}_j}^2_2)^b)^{-1}, \label{eq:wij} % TODO: wie sind a, b gew�hlt?
   % \end{equation}

   % % TODO: f�r a=b=1 tSNE, b=1 LargeVis

   % wobei $\mathbf{y}_i, 1 \leq i \leq N$ die Vektoren der Einbettung sind. Die Wahl der Werte $a,b$ werden wir in 
   % Abschnitt \ref{seq:hyper} beschreiben.


   % Somit erhalten wir folgende Kreuzentropie zwischen den Graphen der hoch- und niedrigdimensionalen 
   % Darstellung der Daten:

   % \begin{alignat}{2}
   %    C(V, W) &= \sumtup{i}{j} \vij \log\left(\frac{\vij}{\wij}\right) + (1-\vij) \log\left(\frac{1-\vij}{1-\wij}\right) \\
   %            &= \begin{aligned}[t]
   %                   &  \sumtup{i}{j} (\vij \log(\vij) + (1-\vij) \log(1-\vij)) \\
   %                   &- \sumtup{i}{j} (\vij \log(\wij)) - \sumtup{i}{j} ((1-\vij) \log(1-\wij))
   %                \end{aligned} \\
   %            &=  C_v - \sumtup{i}{j} (\vij \log(\wij) + (1-\vij) \log(1-\wij)) \label{eq:loss}
   % \end{alignat}

   % Der Term $C_v$ bleibt w�hrend der Minimierung der Funktion bez�glich der $\mathbf{y}_i$ konstant.
   
   % Aufgrund der Wahl einer differenzierbaren Funktion f�r den Zugeh�rigkeitsgrad der 1-Simplizes, 
   % bzw. der Gewichte des niedrigdimensionalen Graphen, l�sst sich nun der Gradient der Zielfunktion (\ref{eq:loss}) herleiten.

   % Mittels der Kettenregel ergibt sich mit $d_{ij} \coloneqq \norm{\mathbf{y}_i - \mathbf{y}_j}_2$:

   % \begin{align}
   % \quad
   %    \deriv{C}{\mathbf{y}_i} &= \sumtup{k}{l} \deriv{C}{w_{kl}} \sumtup{m}{n} \deriv{w_{kl}}{d^2_{mn}} \sumtup{p}{q} \deriv{d^2_{mn}}{d_{pq}} \deriv{d_{pq}}{\mathbf{y}_i} \\
   %                   &= \sumtup{k}{l} \deriv{C}{w_{kl}} \sumtup{m}{n} \deriv{w_{kl}}{d^2_{mn}} \deriv{d^2_{mn}}{d_{mn}} \deriv{d_{mn}}{\mathbf{y}_i} \\
   %                   &= 2 \sumtup{k}{l} \deriv{C}{w_{kl}} \sumtups{m} \deriv{w_{kl}}{d^2_{mi}} \deriv{d^2_{mi}}{d_{mi}} \deriv{d_{mi}}{\mathbf{y}_i} \\
   %                   &= 2 \sumtups{m} \left( \sumtup{k}{l} \deriv{C}{w_{kl}} \deriv{w_{kl}}{d^2_{mi}} \right) \deriv{d^2_{mi}}{d_{mi}} \deriv{d_{mi}}{\mathbf{y}_i}  \label{eq:deriv-2}\\
   %                   &= 4 \sumtups{m} \left( \sumtup{k}{l} \deriv{C}{w_{kl}} \deriv{w_{kl}}{d^2_{mi}} \right) (\mathbf{y}_i - \mathbf{y}_m) \label{eq:deriv-1}
   % \end{align}

   % Bei der Umformung von (\ref{eq:deriv-2}) nach (\ref{eq:deriv-1}) haben wir verwendet, dass $d_{mi}$ die euklidische Norm ist. 
   % F�r Einbettungen in andere metrische R�ume m�sste der Gradient an dieser Stelle entsprechend angepasst werden. Da wir 
   % das UMAP Verfahren im wesentlichen zur Visualisierung im $\R^2$ benutzen ist die Wahl der euklidischen Norm gerechtfertigt.
   % Die �brigen Umformungen ergeben sich aus umordnen und wegfallen der Terme. 

   % (\ref{eq:deriv-1}) l�sst sich weiter umformen, dazu nutzen wir:

   % \begin{align}
   %    \deriv{C}{\wij} &= - \frac{\vij}{\wij} + \frac{1-\vij}{1-\wij} = \frac{\wij - \vij}{\wij (1 - \wij)} \\
   %    \deriv{\wij}{d^2_{ij}} &= -bad_{ij}^{2(b-1)} \wij^2 = - \frac{b}{d_{ij}^2} \wij (1 - \wij)
   % \end{align}

   % Der UMAP Gradient ist also gegeben durch:
   % % TODO: Daniel: Fehlt da nicht noch eine innere Summe aus 4.9?
   % \begin{equation}
   %    \deriv{C}{\mathbf{y}_i} = 4 \sumtups{j} (\wij - \vij) \frac{b}{d_{ij}^2} (\mathbf{y}_i - \mathbf{y}_j)
   %    \label{eq:umapgrad}
   % \end{equation}